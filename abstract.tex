


\chapter{Résumé} 

\fixme{Skal nok bruge lidt fra introduktionen til at starte på}

I teksten her behandler vi rum $Y_{m,n}(\sigma)\subset (\C^m)^n$,
defineret ud fra
en permutation $\sigma\in S_m$, og
kvotienter af disse,
\[ X_{m,n}(\sigma) = Y_{m,n}(\sigma) /\T^n. \]
Disse rum har en masse pæne egenskaber, og vi bruger en del tid på at
undersøge deres symmetrier, at studere deres topologi, og se på hvordan
man kan bevæge sig mellem
rummene. Specielt viser det sig at man kan danne $X_{m,n+1}(\sigma)$ ud
fra $X_{m,n}(\tau)$ ved at betragte de rigtige permutationer
$\tau\in S_m$. Denne observation fører til to vigtige værktøjer til at
studere rummene med. For det første giver det en spektralfølge der gør
at vi kan udregne
kohomologi-grupperne for $X_{m,n+1}(\sigma)$ hvis vi kender dem for
rummene $X_{m,n}(\tau)$, hvilket vi bruger til at udregne alle
kohomologi-grupperne for rummet $X_{3,3}$. For det andet får vi en
stabiliserings-afbildning
\[ s : X_{m,n} \to X_{m,n+1} \]
der gør at vi kan tage en grænseværdi af rummene og ende med det
uendelige rum $X_{m,\infty}$. Dette rum viser vi er ækvivalent til
løkkerummet for den fuldstændige flag-mangfoldighed $\SUT{m}$, hvilket
giver os en måde at studere løkkerummet ud fra de approksimerende rum
$X_{m,n}$, men også tillader os at gå den anden vej og få information
om $X_{m,n}$ ud fra løkkerummet. Denne ækvivalens bliver især
studeret for $m = 3$, og specielt er et af hovedresultaterne i
kapitel \ref{chap:loekker} at vi for $n \geq k$ har en isomorfi mellem
kohomologi-grupperne $H^k(X_{3,n})$ og kohomologi-grupperne 
$H^k(\Omega(\SU_3/\T^2))$.

\chapter{Abstract}

\fixme{Oversæt ovenstående og indsæt her}

%%%%%% Previous version, can probably be reused.

% The following document is part of the midway exam for Ph.D. students
% at Aarhus University. It describes the work I have done on my project
% so far, and
% ends with a small presentation of ideas for future work. Due to the
% limited number of pages, the report is meant more as an overview and
% will occasionally only sketch a proof or argument instead of giving
% all the details.

% I would like to thank Thomas Schmidt for proofreading an earlier
% version of this report, and my advisor Marcel B\"okstedt for his help
% and inspiration during the past two years.

% \subsubsection{Conventions and notation}

% What follows is a brief description of the notation used in the
% report.
% For two topological spaces $X$ and $Y$, we will write $X \cong Y$ to
% denote that $X$ and $Y$ are homeomorphic and $X\simeq Y$ to denote
% that they are homotopy equivalent. Likewise, $f \simeq g$ means that
% the maps $f$ and $g$ are homotopic. For subsets, we will use $A
% \subset X$ when $A$ is any subset of $X$, and we will generally not
% make a distinction between points of $X$ and one-point subsets of
% $X$. The set difference of $X$ and $A$ will be written as $X-A$. With
% a distinguished basepoint $x_0$ in $X$, the set of based loops in $X$
% will be denoted $\Omega(X) = \set{ \g : S^1 \to X \mid \g(1) = x_0}$,
% and this comes equipped with the compact-open topology. The one-point
% compactification of $X$ is $X^+ = X\cup \set{\infty}$.

% We will be working exclusively with topology in the complex numbers,
% so the disc $D$ is the unit disc in the complex plane $\C$ and $D_p$
% is the unit disc in $\C^p$. The non-zero elements in $\C$ will be
% written as $\C^* = \C - 0$. $\GL_m$ will denote the group of
% linear automorphisms of $\C^m$ and the standard basis of $\C^m$ is
% $e_1,\dots,e_m$. Linear maps will be written as matrices with respect
% to the standard basis.

% Since we are mainly concerned with algebraic topology, a space will
% occasionally be replaced with a homotopy equivalent one without
% mention. For example, $T^n$ will denote both the torus $(\C^*)^n
% \subset \GL_n$ and $(S^1)^n\subset \C^n$ depending on context, and the
% punctured disc $D_p-0$ will sometimes be identified with the sphere
% $S^{2p-1} \subset \C^p$. Likewise, we will not distinguish between the
% different models of the flag manifold of $\C^m$, given as $\GLB{m}
% \cong \UT{m} \cong \SUT{m}$ with $B_m \subset \GL_m$ the upper (or
% lower) triangular matrices in $\GL_m$, $T^m \subset \U_m$ the diagonal
% matrices in the unitary group and $T^{m-1}\subset \SU_m$ the diagonal
% matrices in the special unitary group.

% Homology and cohomology groups will always be singular groups
% with integer coefficients. If a theorem is mentioned by name without a
% reference, it can be found in \cite{hatcher}.

% %\vfill % Er allerede så langt nede som muligt.
% \begin{flushright}
%   Simon Stoltze \\
%   Aarhus University\\
%   January, 2015
% \end{flushright}



%%% Local Variables: 
%%% mode: latex
%%% TeX-master: "main"
%%% End: 
