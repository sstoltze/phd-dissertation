\chapter{Preface} 

\fixme{Referer til del a, sig at dele er taget derfra. Se
  http://phd.au.dk/fileadmin/grads.au.dk/ST/Rules\_and\_regulations/GSST\_Rules\_and\_Regulations.pdf,
  \$11.1}


This dissertation is the outcome of my Ph.D. studies at the Department
of Mathematics, Aarhus University, from February 2013 to October
2016. It contains and expands on the subjects covered in my combined
progress report and Master's thesis, which was finished and defended
in January 2015.

I would like to thank my Ph.D. advisor, Marcel B\"okstedt, for both
suggesting this project and helping me with plenty of meetings and long
discussions when I was stuck, as well as providing
encouragement and good advice when it was needed most.

I would also like to thank my office mates and fellow Ph.D. students
at the Department of Mathematics
for both providing valuable feedback on ideas and for keeping me
motivated.

\fixme{Acknowledgements etc. Venner, kolleger, familie, Julie?}

\section{Notation and conventions}

Throughout the dissertation, the following conventions will be used.
For two topological spaces $X$ and $Y$, we will write $X \cong Y$ to
denote that $X$ and $Y$ are homeomorphic, and $X\simeq Y$ to denote
that they are homotopy equivalent. Likewise, $f \simeq g$ means that
the maps $f$ and $g$ are homotopic. For subsets, we will use $A
\subset X$ when $A$ is any subset of $X$, and we will generally not
make a distinction between points of $X$ and one-point subsets of
$X$. The set difference of $X$ and $A$ will be written as $X-A$.

We will be working almost exclusively with topology in the complex
numbers,
so the disc $D\subset\C$ is the unit disc in the complex plane and
$D_p$ is the unit disc in $\C^p$. The non-zero elements in $\C$ will
be written as $\C^* = \C - 0$. $\GL_m(\C)$ will denote the group of
linear automorphisms of $\C^m$ and the standard basis of $\C^m$ is
$e_1,\dots,e_m$,
\[ e_i =
\big(0, \dots, 0, \overset{\substack{i\\\downarrow}}{1}, 0\dots,
0\big).\]
Linear maps will be written as matrices with respect
to the standard basis.

Since we are mainly concerned with algebraic topology, a space will
occasionally be replaced with a homotopy equivalent one without
mention. For example, $\T^n$ will denote both the complex units
$(\C^*)^n$, the
diagonal matrices in $\GL_n(\C)$, and $(S^1)^n$ depending
on context, and
the
punctured disc $D_p-0$ will sometimes be identified with the sphere
$S^{2p-1} \subset \C^p$. Likewise, we will not distinguish between the
different models of the flag manifold of $\C^m$, given as $\GLB{m}
\cong \UT{m} \cong \SUT{m}$ with $B_m \subset \GL_m(\C)$ the upper (or
lower) triangular matrices in $\GL_m(\C)$, $\T^m \subset \U_m$ the
diagonal
matrices in the unitary group and $\T^{m-1}\subset \SU_m$ the diagonal
matrices in the special unitary group.

Homology and cohomology groups will always be singular groups
with integer coefficients. If a theorem is mentioned by name without a
reference, it can be found in \cite{hatcher}.

We will use permutations a great deal. The permutations of the set
$\set{1,\dots,n}$ will be denoted $S_n$.
These will be written as products of cycles, where we multiply by
considering permutations as functions and then composing. For
example, the
permutation $\sigma\in S_3$, given by
\begin{align*}
  \sigma(1) &= 3, \\
  \sigma(2) &= 1, \\
  \sigma(3) &= 2,
\end{align*}
will be written as $\sigma = (1 \; 3 \; 2) = (2\; 3)\cdot (1\; 2)$.
For our purposes, it will be useful to occasionally consider
a permutation as a matrix. We will do this by working with a fixed
basis, $e_1,\dots,e_n$. An element $\sigma \in S_n$ then
corresponds to the matrix defined by
\[ \sigma \cdot e_i = e_{\sigma(i)}. \]
So a permutation multiplied on a matrix from the left permutes the
rows of the matrix.


\vfill
\begin{flushright}
  Simon Stoltze \\
  Aarhus University\\
  October, 2016
\end{flushright}




%%% Local Variables:
%%% mode: latex
%%% TeX-master: "main"
%%% End:
