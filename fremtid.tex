
\chapter{Future work}
\label{chap:fremtid}

I will end with a brief discussion of possible future work that could
be done to study the spaces $X_{m,n}(\sigma)$ and their relation to
the loop space.

\section{Stabilisation}

An obvious first place to start would be to prove Theorem
\ref{thm:kostab} for $m$ greater than $3$, giving something like the
following.

\begin{conjecture}
  For each $m$, there is a function $\lambda$ from the symmetric
  group $S_m$ to the integers, such that the stabilisation map
  \[ L_n^* : H^k(X_{m,\infty}(\sigma)) \to H^k(X_{m,n}(\sigma)) \]
  is an isomorphism for $n \geq k + \lambda(\sigma)$.
\end{conjecture}

The exact definition of $\lambda$ is probably one of the hard parts of
this conjecture. It will need to satisfy something similar to the
inequalities stated in the previous chapter, which would probably look
like
\begin{align*}
  \lambda(\sigma) &\geq \lambda(\sigma_{\emptyset}), \\
  \lambda(\sigma) &\geq \lambda(\sigma_{\set{1}}) - 1,\\
  &\vdots \\
  \lambda(\sigma) &\geq \lambda(\sigma_{\set{m}}) - 1,\\
  &\vdots \\
  \lambda(\sigma) &\geq \lambda(\sigma_{\set{1,\dots,m}}) - m + 1.
\end{align*} \fixme{Overvej dette igen. Og burde nok kun være
  1,...,m-1 i indekset, men kan ikke lige få det formuleret på en
  ordentlig måde}
By considering the function defined on $S_3$ in the previous chapter,
it seems like the definition should depend on $\sigma(1)$, but it may
also be related to the Bruhat ordering on $S_m$, which means it
depends on the minimum number of simple transpositions required to
write $\sigma$ as a product.

The diagram obtained by considering $L_n^*$ as a map of spectral
sequences,
\[ L_n^* : E_*^{*,*}(\infty,\sigma) \to E_*^{*,*}(n+1,\sigma), \]
also played a role in the proof. The conjecture above will probably
also require a study of this diagram in more generality than was
considered here. There is a risk that the spectral sequence for
$X_{m,\infty}(\sigma)$ does not collapse as readily for high $m$ as it
does for $m=3$, in particular because increasing $m$ increases the
number of non-zero columns on the first page of the spectral sequence,
$E_1^{*,*}(\infty,\sigma)$. Likewise, the maps induced from $L_n^*$
will also have to be well-behaved on the different columns of the
spectral sequence if the proof is to carry over without too many
changes. This, in particular, is where the choice of $\lambda$ plays
a crucial role.

\subsection{Inverse stabilisation}

During the study of these spaces, some computer programs were written
to simplify some computations. By continuing to work with these, and
applying some conjectures about the structure of the spectral
sequence, we arrived at a conjecture that relates the groups
$H^k(X_{3,n}(\sigma))$ and $H^k(X_{3,\infty}(\sigma))$ for $k$ larger
than $n$. This is the opposite situation to what was considered in
Theorem \ref{thm:kostab}, so a proof of this would mean that we
would know almost all the cohomology groups of $X_{3,n}(\sigma)$,
without doing any explicit calculations. The conjecture is very loose
at the moment, but would probably look something like the following.
\begin{conjecture}
  Blah \fixme{Find ud af hvad der skal være her}
\end{conjecture}

And like above, if this could be established it would seem natural to
consider $m > 3$ and try to prove a similar theorem for
$X_{m,n}(\sigma)$.

\subsection{Other coefficients}

In a similar fashion, it seems very natural to try and prove the
stabilisation theorem for the other possible
coefficient-groups. Nothing in the proof seems to rely on the
coefficients being complex numbers, so the proof should be almost
automatic if a suitable $\lambda$ could be defined.

\section{Generalised spaces}

Another avenue of study could be to consider all the spaces
$X_{m,n}(\sigma)$ at the same time:
\[ \widehat{X}_{m,n} = \bigcup_{A \in \GL_m(\C)} X_{m,n}(A,A). \]
Since the space $X_{m,n}$ approximates the loop group when $n$ is
large, by having a sequence of vectors map to a loop at the identity,
one could expect $\widehat{X}_{m,n}$ to approximate the free loop
group of all maps from the circle to $\GL_m(\C)$, by mapping a
sequence of vectors starting and ending at $A$ to a loop at $A$. This
would require a study of similarly defined stabilisation maps for
these spaces and seeing what happens when we take a limit. This
is summarised in the following conjecture.

\begin{conjecture}
  There are suitable stabilisations maps
  \[ s : \widehat{X}_{m,n} \to \widehat{X}_{m,n+1}, \]
  so that the space
  \[ \widehat{X}_{m,\infty} = \varinjlim_n \widehat{X}_{m,n} \]
  is homotopy equivalent to the free loop space
  \[ \mathrm{L}(\GL_m(\C)) = \set{ \gamma : S^1 \to \GL_m(\C) }. \]
\end{conjecture}

Like the spaces $X_{m,n}$ considered here, this would give another way
to study a complicated space, $\mathrm{L}(\GL_m)$, by working with
simpler versions that consist of a finite sequence of points that we
connect with nice paths.

\fixme{En eller anden form for afslutning...}

% Slå sammen med sidste kapitel?

% Se bagside af blok. Blah blah blah
% Formodning om at s^* er surjektiv.
% Erstat \C med \R eller \H
% Se om rummene A....A approksimerer det frie løkkerum

%%% Local Variables: 
%%% mode: latex
%%% TeX-master: "main"
%%% End: 
