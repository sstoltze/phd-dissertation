\chapter{Introduction}

\fixme{Skriv det der skal være før her. Snak med Marcel. Og få
  indledningen til at give mening}

The spaces we will be workng with were originally introduced
as... blah blah, nulsnit af v.b., et eller andet, blah, matricer.


The spaces $X_{m,n}(\sigma)$ that we will be studying in the thesis
were originally studied in a slightly simpler form by my advisor,
Marcel B\"okstedt. He looked at sequences of vectors $(v_1,\dots,v_n)$
in $\C^2$, with the requirement that $v_i$ and $v_{i+1}$ are linearly
independent, that the first entry of $v_1$  is non-zero, and either
the first or the second entry of $v_n$ is non-zero as well. Using the
notation introduced in this thesis, these are the spaces $Y_{2,n}$ or
$Y_{2,n}((1\;2))$ from Definition \ref{def:rum}. Taking a quotient
of the action
\begin{align*}
  \T^n \times Y_{2,n}(\sigma) &\to Y_{2,n}(\sigma) \\
  (\l_1,\dots,\l_n)\cdot (v_1,\dots,v_n) &= (\l_1v_1,\dots,\l_nv_n),
\end{align*}
we end up with the spaces referred to as $X_{2,n}(\sigma)$ in
Definition \ref{def:kvotientrum}. He then showed that $Y_{2,n}$ was
covered by the spaces $Y_{2,n-1}\times\T^2$ and $Y_{2,n-1}((1\; 2))\times
\T$, giving inclusion maps
\[ s : Y_{2,n-1} \cong Y_{2,n-1}\times\set{(1,1)} \to Y_{2,n}. \]
By using the Mayer-Vietoris sequence for the cover, and the
corresponding cover of $X_{2,n}$, he proved the
following theorem.
\begin{theorem}
  The space $X_{2,n}$ has homology groups
  \[ H_q(X_{2,n}) \cong
  \begin{cases}
    \Z & 0 \leq q \leq n, \\
    0 & \text{otherwise.}
  \end{cases} \]
  The inclusion
  \[ s : X_{2,n-1} \to X_{2,n} \]
  is an isomorphism on homology when the degree is $\leq n-1$.
\end{theorem}

The homology of $X_{2,n}$ consists of a copy of the integers in all
degrees up to $n$, so when $n$ is large it looks like the homology
groups of the loop space
$\Omega(\SU_2/\T)$, which has a copy of the integers in each
degree. We can define a map from $X_{2,n}$ to the loop space
by sending $(v_1,\dots,v_n)$ to
the path given by connecting up the matrices
\[ \Id \leadsto [e_2,v_1] \leadsto [v_1,v_2] \leadsto \dots \leadsto
\Id, \]
and it turns out that this map is an isomorphism on homology when the
groups are non-zero.

The aim of this dissertation is to extend some of
these results to $X_{m,n}$, $m > 2$, with a focus on cohomology
instead of homology. The cohomology groups become much harder to
calculate when $m > 2$, since the Mayer-Vietoris sequence that was
used for $X_{2,n}$
does not give enough information. Instead we have to cover $X_{m,n}$
with several subspaces, which means we have to use spectral
sequences to compute the cohomology groups and this leads to increased
complexity. However, it is still possible to prove that $X_{m,n}$
approximates the loop space $\Omega(\SUT{m})$ when $n$ is large, which
is the content of Theorem \ref{thm:loekker}. This approximation allows
us to compute $H^q(X_{3,n})$ when $n \geq q$, as is done in Theorem
\ref{thm:kostab}.

The thesis starts with a short introduction to some of the tools and
subjects that
will be required in the following chapters, contained in Chapter
\ref{chap:generelt}. Each section consists of a brief description of
a subject, with references to
important theorems needed for our purposes. The topics that are
covered are spectral
sequences, loop spaces, flags, and Coxeter groups, and since it is not
possible to give a full overview of any of these in a couple of
pages, every section also contain references to literature with a
more extensive coverage.

In Chapter \ref{chap:rum}, we give a description of the spaces we will
be working with, first introducing subspaces $Y_{m,n}(\sigma)$ of
$(\C^{m})^{n}$, which depend on permutations $\sigma\in S_m$, and
then quotienting out with an action of the complex torus $\T^n =
(\C^*)^n$, giving the spaces $X_{m,n}(\sigma)$ that we will mostly be
working with. We show that these spaces are non-empty when $n$ is
large enough, and we spend some time working out how the space
$Y_{m,n}(\sigma)$ can be built from the spaces $Y_{m,n-1}(\tau)$, and
exactly which permutations $\tau$ are necessary to do this in Theorem
\ref{thm:permutation}. This
allows us to take a limit, and gives us the spaces
$Y_{m,\infty}(\sigma)$ and $X_{m,\infty}(\sigma)$. We end
with a very brief discussion of the spaces we get when we replace the
complex numbers with the reals or quaternions.

Chapter \ref{chap:udregninger} starts with a description of a spectral
sequence that will be used extensively. This allows us to compute the
cohomology of the space $Y_{m,n}(\sigma)$ from the cohomology of the
spaces $Y_{m,n-1}(\tau)$ that it is built from. We immediately use
this spectral sequence to work out the first cohomology group of both
$Y_{m,n}$ and $X_{m,n}$ for all $m$ and $n$. The rest of the
chapter is spent doing computations with the spectral sequence for the
spaces $Y_{3,3}$ and
$X_{3,3}$, ending with a complete description of the cohomology
groups for these spaces.

The following chapter, Chapter \ref{chap:loekker}, studies the limit
spaces $X_{m,\infty}(\sigma)$ in detail. We start by showing that they
are homotopy equivalent to the loop space $\Omega(\SUT{m})$, which
gives us the cohomology of the limit space. The proof does not depend
on working over the complex numbers, so it easily carries over to the
real and quaternionic cases. We then give a brief
mention of other ways we could define the limit space, which are all
equivalent from a homotopy theoretic standpoint. By working
with one of these other stabilisations of $X_{3,n}(\sigma)$, we get
Theorem \ref{thm:kostab}
that tells us when the cohomology groups of
$X_{3,n}(\sigma)$ stabilise to the cohomology groups of
$X_{3,\infty}(\sigma)$.

The thesis concludes with a brief description of the open questions
still remaining, contained in Chapter \ref{chap:fremtid}. These are
questions that have been raised during the last parts of the project
and thus have not been studied in detail, as well as small ideas for
how the results could be applied to the free loop space instead of
the space of based loops considered in the thesis.



% En introduktion til afhandlingen, inklusive kapitel-oversigt og
% ideer bag. Her bør for eksempel indgå en forklaring af det der pjat
% med vektorbundter som Marcel har forsøgt at forklare før.
% Derudover nok også en oversigt over de vigtigste resultater der
% indgår.

%%%%% Previous
% \chapter{Introduction}

% % Noget om starten, hvorfor dette er interessant. Se bagside af blok
% % Noget om notation... eg \subset, D-0, hvad er D_p
% % Alting komplekst, så \GL_m giver mening etc. Måske også de
% % forskellige versioner af flag-mangfoldigheder.
% % Og noget med at T=\C^*....

% This document gives a description of a family of spaces $\X\subset
% \C^{mn}$, indexed by two natural numbers $m$ and $n$, and a quotient
% of these spaces, $Y_{m,n} = \X / T^n$. The spaces were
% originally defined as the zero-set of a section on a vector bundle of
% a Grassmann manifold, and were studied to understand the properties of
% families of Calabi-Yau manifolds. However, the spaces turned out to
% have other properties that made them interesting in their own
% right. The most important property is that they can be used to form a
% space that is homotopy equivalent to a space that appears in physics,
% namely the loop space $\Omega(\SUT{m})$. This hints at a way of
% extracting information about the loop space by working with the
% simpler spaces $Y_{m,n}$.

% The spaces have been studied by Marcel B\"okstedt, and some of his
% results are gathered in \cite{bokstedt}. In particular it includes the
% following theorem that covers the special case where $m=2$:
% \begin{theorem}
%   The homology of $Y_{2,n}$ is
%   \[ H_q(Y_{2,n}) =
%   \begin{cases}
%     \Z & 0 \leq q \leq n \\
%     0 & \text{otherwise}
%   \end{cases} \]
%   There is a stabilization map $s : Y_{2,n} \to Y_{2,n+1}$ that
%   induces an isomorphism
%   \[ s_* : H_q(Y_{2,n}) \to H_q(Y_{2,n+1}) \]
%   for $q \leq n$. There is also a map $f : Y_{2,n} \to
%   \Omega(\mathrm{SU_2}/T^1)$ that induces an isomorphism in homology in
%   degree $q\leq n$.
% \end{theorem}

% The aim of the following is to try and extend these results to the
% general case of $Y_{m,n}$, with $m\geq 2$.

% This progress report is divided into three main parts.
% The first chapter gives an introduction to the spaces we are
% interested in and lists some of the properties that will be used
% in the other chapters.
% The second chapter is mainly concerned with an explicit calculation of
% the cohomology of the spaces $X_{3,3}$ and $Y_{3,3}$. This is done by
% applying the spectral sequence coming from a filtration of $X_{3,3}$,
% computing the various relative cohomology groups needed, and
% finding the differentials of the sequence.
% The third and final chapter concerns the relation to the loop space
% described above. Here it is shown that by taking a limit of the spaces
% $Y_{m,n}$, we get a space homotopy equivalent to the loop space. The
% chapter concludes with a brief overview of plans for future studies.

%%% Local Variables: 
%%% mode: latex
%%% TeX-master: "main"
%%% End: 
