
\chapter{Sage code} 
\label{appendix:sage}

The Sage code that was used for computations, referenced in Chapter
\ref{chapter:udregninger}, is included below. Note that a red arrow at
the beginning of a line of code denotes a linebreak that was inserted
to make the code fit on the page and is not there in the original
code.

\section{The boundary map}
\label{sec:sageboundary}

The code below computes the kernel and image of the first boundary map
in the spectral sequence for $Y_{3,3}$, as described in Section
\ref{sec:rand}. It does so by creating a ring with the correct
generators and relations and encoding the boundary map in two steps,
firstly using the inclusion of the relevant spaces and then using the
boundary map from the pair $(D,D-0)$. It the splits the cohomology
groups appearing in the spectral sequence into ranks and computes
kernel and image for the boundary. To get a readable output, load the
Sage file and call either the function \lstinline{pretty_print} for
the full $E_2$-page of $Y_{3,3}$ or \lstinline{print_mod} for the
cohomology of $X_{3,3}$.

\fixme{Skriv kommentarer i koden}
\lstinputlisting[language=Python]{./sage/rand.sage}


\section{Orientation of intersections}
\label{sec:sageorientation}

The code below computes the orientation of the points of intersection
of the inclusion maps, as described in Section
\ref{sec:orientation}. Note that the code is old, and hence uses a
slightly different naming convention than is otherwise used in this
thesis. In particular, the space $X_{001}$ in the code corresponds to
the space $Y_{\set{1,2}}$ in the thesis, $X_{101}$ corresponds to
$Y_{\set{1}}$, $X_{011}$ to $Y_{\set{2}}$ and $X_{111}$ to $Y_{\emptyset}$.

\lstinputlisting[language=Python]{./sage/orientation.sage}


%%% Local Variables: 
%%% mode: latex
%%% TeX-master: "main"
%%% End: 
