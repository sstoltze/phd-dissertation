% Her skal samles lidt generel information om SS og løkkerum

\chapter{Spectral sequences and loop spaces}
\fixme{Bedre titel...}

The purpose of this chapter is to give a brief introduction to two
important topics in this thesis, namely spectral sequences and loop
spaces. Spectral sequences are an important tool in algebraic topology
and can be considered a generalisation of long exact sequences, while
loop spaces are a class of spaces that appear naturally when
considering the spaces that will be studied in this thesis. The
following pages will collect some basic info and state the results
that are of interest in next chapters, while providing references for
proofs and further information. \fixme{Måske noget med at vi starter
  ca. fra Hatcher, alt andet betragtes som ``nyt''}

\section{Spectral sequences}
\label{sec:ss}

A spectral sequence, as we will be using them, is a tool for computing
the cohomology groups of a topological space, $X$, by considering an
increasing sequence of subspaces
\[ X_0 \subset X_1 \subset X_2 \subset \dots \subset X, \]
with the union covering all of $X$,
\[ X = \bigcup_{i} X_i. \]
This is a generalisation of the long exact sequence of a pair, where
we only have a single subspace $A \subset X$. Another simple example
that most people are familiar with is if there are two subspaces,
\[ X = A \cup B, \]
in which case the cohomology of $X$ can often be computed from the
cohomology of $A$, $B$ and $A\cap B$ by using the Mayer-Vietoris
sequence. This likewise corresponds to a spectral sequence to compute
the cohomology of $X$ by working with an abitrary covering. As might
be expected, since spectral sequences work more generally there is
also a corresponding increase in complexity when using them for
calculations. In the following we will focus on the specific case that
we will be using in this text. For a general introduction to the
subject, see either \cite{hatcherss} or \cite{mccleary}.

\subsection{The spectral sequence of a filtration}
\label{sec:ss-filtration}

Consider a space $X$ with a filtration, that is an increasing sequence
of subspaces that cover $X$:
\[ X_0 \subset X_1 \subset X_2 \subset
\dots \subset X_k  = X. \]
All of the following can be modified to work if the filtration is not
finite, but we will not need this here. We will augment the filtration
by defining additional subspaces,
\[ X_s =
\begin{cases}
  \emptyset & s \leq -1 \\
  X & s \geq k+1
\end{cases}. \]

With such a sequence, we can
consider the relative cohomology groups
\[ E_1^{p,q} = H^{p+q}(X_p,X_{p-1}), \]
and the differentials between them coming from the long exact sequence
of the triple $(X_{p+1},X_p,X_{p-1})$:
\[ d_1 : E_1^{p,q} = H^{p+q}(X_p,X_{p-1}) \to H^{p+q+1}(X_{p+1},X_p) =
E_1^{p+1,q}. \]
This map factors as the composition of an
inclusion of the pair $(X_p,X_{p-1})$ and the boundary map of the pair
$(X_{p+1},X_p)$,
\[ \xymatrix{ d_1 : H^{p+q}(X_p,X_{p-1}) \ar[r]^<<<<<{\jmath^*} &
  H^{p+q}(X_p) \ar[r]^<<<<<{\delta} & H^{p+q+1}(X_{p+1},X_p),
} \]
and in particular we get that the composition of two such maps must be
zero since the composition $\delta \circ \jmath^* = 0$ by
exactness. So we can compute the homology of these groups and get new
groups,
\[ E_2^{p,q} = \ker d_1 / \im d_1. \]
More suprisingly, we can also find new maps
\[ d_2 : E_2^{p,q} \to E_2^{p+2,q-1} \]
that make these new groups into a chain complex, allowing us to form
\[ E_3^{p,q} = \ker d_2 / \im d_2. \]
This process can be continued, giving a sequence of groups and maps
\[ \xymatrix{d_r : E_r^{p,q} \ar[r] & E_r^{p+r,q-r+1} } \]
where the composition $d_r \circ d_r$ is zero and
$E_r^{p,q}$ is formed by taking the homology of $E_{r-1}^{p,q}$ with
the map $d_{r-1}$. Since we have $E_r^{p,q} = 0$ for $p < 0$ or $p >
k$, these maps will eventually always go either to or from a zero
group and will thus not depend on the value of $r$. This allows us to 
define the stable groups $E_\infty^{p,q}$ as these independent
groups. The following theorem relates them to the groups $H^*(X)$ that
we are interested in calculating.

\begin{theorem}
  \label{thm:ss}
  Given a filtration $\set{X_s}$ of a topological space $X$ as above,
  there is a sequence of abelian groups $E_r^{p,q}$ and maps
  \[ d_r : E_r^{p,q} \to E_r^{p+r,q-r+1} \]
  that satisfy the following:
  \begin{itemize}
  \item The first groups are
    \[ E_1^{p,q} = H^{p+q}(X_p,X_{p-1}) \]
    and the map $d_1$ is the boundary map of the triples
    $(X_{p+1},X_p,X_{p-1})$.
  \item The composition $d_r \circ d_r$ is zero and
    \[ E_{r+1}^{p,q} = \ker d_r / \im d_r, \]
    computed at $E_r^{p,q}$.
  \item There is a filtration of the $n$'th cohomology group of $X$,
    \[ 0 \subset F_n^n \subset \dots F_0^n = H^n(X), \]
    with the limit groups $E_\infty^{p,n-p}$ isomorphic to the
    quotients
    \[ E_\infty^{p,n-p} \cong F_p^n / F_{p+1}^n. \]
  \end{itemize}
\end{theorem}

This theorem is a combination of Proposition 1.2 and Theorem 1.14 in
\cite{hatcherss}, and the proof can be found there as well. Assuming
we can compute the spectral sequence, for which there are several
tricks, the groups $H^n(X)$ can then be found by solving some
extension problems of the form
\[ 0 \to F_{p+1}^n \to F_p^n \to E_\infty^{p,n-p} \to 0, \]
starting from the short exact sequence
\[ 0 \to E_\infty^{n,0} \to F_{n-1}^n \to E_\infty^{n-1,1}. \]
This can for example be done if $E_{\infty}^{p,n-p}$ is free for all
$p$, in which case $H^n(X)$ is just the direct sum of these groups,
\[ H^n(X) \cong \bigoplus_p E_\infty^{p,n-p}. \]

\section{Loop spaces}
\label{sec:ls}





%%% Local Variables: 
%%% mode: latex
%%% TeX-master: "main"
%%% End: 
