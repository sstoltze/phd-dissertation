% Her skal samles lidt generel information om SS og løkkerum

\chapter{Preliminaries}
\label{chap:generelt}

The purpose of this chapter is to give a brief introduction to some
important topics, most importantly spectral sequences
and loop spaces. Spectral sequences are an important tool in algebraic
topology and can be considered a generalisation of long exact
sequences, while
loop spaces are a class of spaces that appear naturally when
considering the spaces that will be studied in this thesis. The
following pages will collect some basic info and state the results
that are of interest in next chapters, while providing references for
proofs and further information. The chapter also includes a very brief
introduction to flags and Coxeter groups, to give enough
information for the small part these two subjects will play in the
main text.

\section{Spectral sequences}
\label{sec:ss}

A spectral sequence, as we will be using them, is a tool for computing
the cohomology groups of a topological space, $X$, by considering an
increasing sequence of subspaces
\[ X_0 \subset X_1 \subset X_2 \subset \dots \subset X, \]
with the union covering all of $X$,
\[ X = \bigcup_{i} X_i. \]
An example that most people should be familiar with is the long exact
sequence of a pair, where
we have a single subspace $A \subset X$ and try to compute the
cohomology of $X$ from the cohomology of $A$ and the cohomology of the
pair $(X,A)$. Another simple example
is the case where there are two subspaces,
\[ X = A \cup B, \]
in which case the cohomology of $X$ can often be computed from the
cohomology of $A$, $B$ and $A\cap B$ by using the Mayer-Vietoris
sequence. This generalises to a spectral sequence to compute
the cohomology of $X$ by working with an abitrary covering. As might
be expected, since spectral sequences work more generally they also
carry a corresponding increase in complexity when using them for
calculations. In the following section we will focus on the specific
case that 
we will be using in this text. For a general introduction to the
subject, see either \cite{hatcherss} or \cite{mccleary}.

\subsection{The spectral sequence of a filtration}
\label{sec:ss-filtration}

Consider a space $X$ with a filtration, an increasing sequence
of subspaces that cover $X$:
\[ X_0 \subset X_1 \subset X_2 \subset
\dots \subset X_k  = X. \]
All of the following can be modified to work if the filtration is not
finite, but we will not need this here. We will augment the filtration
by defining additional subspaces,
\[ X_s =
\begin{cases}
  \emptyset & s \leq -1, \\
  X & s \geq k+1.
\end{cases}. \]

With such a sequence, we can
consider the relative cohomology groups
\[ E_1^{p,q} = H^{p+q}(X_p,X_{p-1}), \]
and the differentials between them coming from the long exact sequence
of the triple $(X_{p+1},X_p,X_{p-1})$:
\[ d_1 : E_1^{p,q} = H^{p+q}(X_p,X_{p-1}) \to H^{p+q+1}(X_{p+1},X_p) =
E_1^{p+1,q}. \]
This map factors as the composition of an
inclusion of $X_p$ into the pair $(X_p,X_{p-1})$ and the boundary map
of the pair
$(X_{p+1},X_p)$,
\[ \xymatrix{ d_1 : H^{p+q}(X_p,X_{p-1}) \ar[r]^<<<<<{\jmath^*} &
  H^{p+q}(X_p) \ar[r]^<<<<<{\delta} & H^{p+q+1}(X_{p+1},X_p),
} \]
and in particular we get that the composition of two such maps,
\[ d_1 \circ d_1 = \delta \circ \jmath^* \circ \delta \circ
\jmath^*, \]
must be
zero since the composition $\jmath^*\circ\delta$ is zero by the
long exact seuqence of the pair $(X_{p+1},X_p)$. So we can compute the
homology of these groups and get new groups,
\[ E_2^{p,q} = \ker d_1 / \im d_1. \]
More surprisingly, we can also find new maps
\[ d_2 : E_2^{p,q} \to E_2^{p+2,q-1} \]
that make these new groups into a chain complex, allowing us to form
\[ E_3^{p,q} = \ker d_2 / \im d_2. \]
This process can be continued, giving a sequence of groups and maps
\[ \xymatrix{d_r : E_r^{p,q} \ar[r] & E_r^{p+r,q-r+1} } \]
where the composition $d_r \circ d_r$ is zero and
$E_r^{p,q}$ is formed by taking the homology of $E_{r-1}^{p,q}$ with
respect to the map $d_{r-1}$. Since we have $E_r^{p,q} = 0$ for $p <
0$ or $p > k$, these maps will eventually always go either to or from
a zero
group when $r > k$, and hence the groups $E_r^{p,q}$ will not change
when we
increase $r$. This allows us to
define the stable groups $E_\infty^{p,q}$ as these $r$-independent
groups. The following theorem relates them to the cohomology groups of
$X$ that we are interested in calculating.

\begin{theorem}
  \label{thm:ss}
  Given a filtration $\set{X_s}$ of a topological space $X$ as above,
  there is a sequence of abelian groups $E_r^{p,q}$ and maps
  \[ d_r : E_r^{p,q} \to E_r^{p+r,q-r+1} \]
  that satisfy the following:
  \begin{itemize}
  \item The first groups are
    \[ E_1^{p,q} = H^{p+q}(X_p,X_{p-1}) \]
    and the map $d_1$ is the boundary map of the triples
    $(X_{p+1},X_p,X_{p-1})$.
  \item The composition $d_r \circ d_r$ is zero and
    \[ E_{r+1}^{p,q} = \ker d_r / \im d_r, \]
    computed at $E_r^{p,q}$.
  \item There is a filtration of the $n\!$'th cohomology group of $X$,
    \[ 0 \subset F_n^n \subset \dots \subset F_0^n = H^n(X), \]
    with the limit groups $E_\infty^{p,n-p}$ isomorphic to the
    quotients
    \[ E_\infty^{p,n-p} \cong F_p^n / F_{p+1}^n. \]
  \end{itemize}
\end{theorem}

This theorem is a combination of Proposition 1.2 and Theorem 1.14 in
\cite{hatcherss}, and the proof can be found there as well. Assuming
we can compute the spectral sequence, the group $H^n(X)$ can then be
found by solving extension problems of the form
\[ 0 \to F_{p+1}^n \to F_p^n \to E_\infty^{p,n-p} \to 0, \]
starting from the short exact sequence
\[ 0 \to E_\infty^{n,0} \to F_{n-1}^n \to E_\infty^{n-1,1} \to 0. \]
For example, if we have that the groups $E_{\infty}^{p,n-p}$ are free
for all $p$, the short exact sequences always split and $H^n(X)$ is
isomorphic to the direct sum of these groups,
\[ H^n(X) \cong \bigoplus_p E_\infty^{p,n-p}. \]

\section{Loop spaces}
\label{sec:ls}

Throughout this section, every topological space will have a
distinguished basepoint. We will be looking a spaces of loops,
consiting of maps from the circle to a given space $X$, preserving the
basepoint. For a general introduction to loop spaces, consult
e.g. \cite[Part~3]{milnor} or \cite[Section~8.2]{may}. The results in
this section are taken from \cite{milnor} and the proofs can be found
there as well.

\begin{definition}
  If $X$ is a topological space with basepoint $*\in
  X$, the loop space $\Omega(X)$ is the space of based maps from the
  cicle to $X$,
  \[ \Omega(X) = \set{ \g : (S^1,1)\to (X,*) } = \set{ \g : [0,1] \to X
    \mid f(0) = f(1) = * }. \]

  We equip this set with the compact-open topology, so for a compact
  set $K \subset S^1$ and an open set $U \subset X$, we get an open
  subset of the loop space consisting of functions mapping $K$ to $U$,
  \[ V(K,U) = \set{ \g\in \Omega(X) \mid \g(K) \subset U }. \]
  The collection of all such sets form a subbase for the compact-open
  topology on $\Omega(X)$.
\end{definition}

We will only worry about the case where $X$ is a Riemannian manifold,
so it comes equipped with a Riemannian metric $g$ and an induced
topological metric $d$. In this case, the compact-open topology is
induced by the metric
\[ d^*(\g,\g') = \max_t d(\g(t),\g'(t)), \]
so two loops are close in $\Omega(X)$ when the points $\g(t)$ and
$\g'(t)$ are close for each $t$.
By \cite[Theorem 17.1]{milnor}, this is homotopy equivalent to the
space of piecewise smooth loops in $X$, and we will generally work
with this space instead. The proof will not be repeated here, but
there is a construction in the proof that deserves a mention. For an
open cover $\set{X_\alpha}_{\a\in I}$ of $X$, we define subsets of the
loop space by requiring that the loops are piecewise contained in a
set in the cover,
\[ \Omega_k(X) = \set{\g \in \Omega(X) \delim \Big\vert\delim
  \text{For each } j \leq 2^k, \g_{\left|\left[
        \frac{j-1}{2^k},\frac{j}{2^k} \right]\right.} \text{ is 
    contained in $X_\alpha$ for some $\alpha$.} }. \]
This gives an increasing sequence of open subsets of $\Omega(X)$
that cover the entire space. In particular, a compact subset of
$\Omega(X)$ will be contained in $\Omega_k(X)$ for some $k$. By
defining a cover of $X$ with good properties, we will be able to work
exclusively with loops where we have some control
over their behaviour.


% Nævne at vi ikke behøver køre * -> *, men også kan køre p -> q

Note that instead of considering loops, we could also consider the
space of paths between two points $p,q\in X$, given by
\[ \Omega(X,p,q) = \set{ \g : [0,1] \to X \mid \g(0) = p, \g(1) = q
}. \]
But by composing such a path with a fixed path $\a$ from $q$ to $p$,
we end up with a loop at $p$. If we compose a loop at $p$ with the
path from $p$ to $q$ given by running $\a$ in reverse, we get a path
from $p$ to $q$, and the composition of these two operations is
homotopic to the identity. This shows that all such spaces 
are homotopy equivalent, so we will not make a major distinction
between them.

\section{Flags}
\label{sec:flags}

To ease some arguments, we will need the concept of a flag in a finite
dimensional vector space. In an
$m$-dimensional vector space $V$, a flag is an increasing sequence of
subspaces of $V$,
\[ V_1 \subset V_2 \subset \dots \subset V_m = V, \]
with the dimension of $V_i$ equal to $i$. So the dimension of each
subspace is one greater than the last. Note that if we pick a non-zero
vector $v_1$ in $V_1$, another (non-zero) vector $v_2$ in $V_2 - V_1$,
and continue picking $v_i$ in $V_i - V_{i-1}$, we end up with $m$
linearly independent vectors. This allows us to construct a matrix in
$\GL(V)$ as
\[ [v_1,\dots,v_m], \]
and we can recover the flag from the matrix by taking
\begin{align*}
  V_1 =\, &\spa(v_1), \\
  V_2 =\, &\spa(v_1,v_2), \\
  \vdots\,\, & \\
  V_m =\, &\spa(v_1,\dots,v_m).
\end{align*}
Likewise, we could take the span of the columns starting from $v_m$
and working backwards to get another flag.
\begin{definition}
  Given an invertible matrix $A = [v_1,\dots,v_m]$, the \textit{left
    flag} of $A$ is
  \begin{alignat*}{5}
    \mathrm{Fl_L}(A)\, &&= \Big(&\spa(v_1) &&\subset \spa(v_1,v_2)
    &&\subset \dots &&\subset \spa(v_1,\dots,v_m) = V\Big).
  \end{alignat*}
  The \textit{right flag} of $A$ is defined similarly as
  \begin{alignat*}{5}
    \mathrm{Fl_R}(A)\, &&= \Big(&\spa(v_m) &&\subset
    \spa(v_{m-1},v_m) &&\subset \dots &&\subset
    \spa(v_1,\dots,v_m) = V\Big). 
  \end{alignat*}
\end{definition}
Note that we could multiply any column of $A$ by a non-zero scalar and
we would end up with the same flag. Likewise, if we replace $v_j$ by
$v_j + \lambda v_i$ for a scalar $\lambda$ and indices $j > i$, we do not
change the left flag of $A$. The right flag is unchanged by doing
the same operation with $j < i$. In particular, these opeartions
correspond to multiplying with a lower-triangular matrix $L$ for the
right flag, or an upper-triangular matrix $U$ for the left flag,
giving the identities
\begin{align*}
  \mathrm{Fl_R}(L\cdot A) &= \mathrm{Fl_R}(A), \\
  \mathrm{Fl_L}(A \cdot U) &= \mathrm{Fl_L}(A).
\end{align*}
The cases that will be of most importance to us is when $A$ is a
permutation matrix. Writing these out for future use, we get
\[  \mathrm{Fl_R}(L) = \mathrm{Fl_R}(\Id) \]
when $A$ is the identity, and
\[  \mathrm{Fl_L}(\sigma \cdot U) = \mathrm{Fl_L}(\sigma) \]
for a general permutation $\sigma\in S_m$.
% $L$  is a
% lower-triangular invertible matrix we can do the above operations one
% column at a time, starting from the last, and see that
% \[ \mathrm{Fl_R}(L) = \mathrm{Fl_R}(\Id). \]
% And similarly, if $U$ is upper-triangular and invertible and $\sigma$
% is a permutation matrix:
% \[ \mathrm{Fl_L}(\sigma U) = \mathrm{Fl_L}(\sigma). \]
% This is slightly harder to see, but $\sigma$ permutes the rows of $U$
% by moving the first row to row $\sigma(1)$, the second row to row
% $\sigma(2)$, and so on. By adding the first column to remaining ones,
% with a suitable scalar multiplied on, we can ensure that the only
% non-zero entry in row $\sigma(1)$ of $\sigma U$ is in the first
% column, row $\sigma(1)$, just as it is in the matrix
% $\sigma$. Continuing like this gives the desired identity.

\section{Coxeter groups}
\label{sec:bruhat}

The final subject to be introduced is Coxeter groups and the Bruhat
order. For a proper introduction, consult either \cite{bjorner} or
\cite{hiller}.

\begin{definition}
  A \textit{Coxeter group} is a group $W$ along with a set of
  generators $S$,
  where all $s \in S$ have order two. The generators are only allowed
  to have relations of the form
  \[ (s s')^{m(s,s')} = \Id \]
  for some function $m : S \times S \to \N\cup\set{\infty}$, with
  $m(s,s) = 1$ and $m(s,s') \geq 2$ for $s\neq s'$.
\end{definition}

An important property of Coxeter groups is that any element $w \in W$
can be written as a \textit{reduced product},
\[ w = s_1 s_2 \dots s_k, \]
where $k = \ell(w)$ is minimal. This number is called the
\textit{length} of $w$. It is a theorem, see for example \cite[Chapter
1.5]{bjorner}, that any two reduced products of $w$ can be obtained
from each other by using \textit{braid moves} of the form
\[ s s' s s' \dots s s' = s' s s' s \dots s' s, \]
where each side of the equality has $m(s,s')$ elements. Another
important property is that any expression of an element $w$ as a
product of generators contains a reduced product for $w$, obtained by
deleting an even number of the generators. With more
work, it is possible to show that products of generators are unique up
to using braid moves and inserting or deleting the identity in the
product as $s^2$.

Our primary example of a Coxeter group will be the
symmetric group $S_m$.
The generators of $S_m$ as a Coxeter group are the \textit{simple
  transpositions}, which are the elements
\[ \set{(1\; 2), (2\; 3), \dots, (m-1 \; m) }. \]
The length of $\sigma$ is the minimal number of simple transpositions
needed to write $\sigma$, which is the same as the number of
\textit{inversions} of $\sigma$,
\[ \ell(\sigma) = \size{\set{ (i,j) \mid 1\leq i < j\leq m, \sigma(i) >
  \sigma(j)}}. \]

Coxeter groups show up in various places, but we will require two
properties. The first is that the symmetric group is the Weyl group of
$\GL_m(\C)$, which for our purposes means that it is a Coxeter group
and if we let $\B^U$ denote the upper-triangular matrices in
$\GL_m(\C)$, there is a \textit{Bruhat decomposition},
\[ \GL_m(\C) = \bigcup_{\sigma \in S_m} \B^U \cdot \sigma \cdot \B^U. \]
Note that this is a disjoint union. For a proof of this, see
\cite[Theorem 4.3]{hiller}. We
would prefer to change this slightly, which is done by considering the
element $\sigma_0$ in $S_m$ defined by
\[ \sigma_0(k) = m+1-k. \]
This element is its own inverse, which gives us
\begin{align*}
  \GL_m(\C) = \sigma_0 \cdot \GL_m(\C) &= \bigcup_{\sigma\in S_m}
                                         \sigma_0\cdot \B^U \cdot
                                         \sigma\cdot \B^U \\
                                       &= \bigcup_{\sigma\in S_m}
                                         \sigma_0\cdot \B^U \cdot
                                         \sigma_0^2\cdot \sigma \cdot
                                         \B^U \\
                                       &= \bigcup_{\sigma\in S_m}
                                         \B^L\cdot \sigma_0\cdot
                                         \sigma\cdot \B^U.
\end{align*}
This is essentially the same theorem, but we have replaced the first
copy of the upper-triangular matrices $\B^U$ with lower-triangular
matrices $\B^L$, by
conjugating with $\sigma_0$.

The second property is that Coxeter groups come equipped with a
partial order, called the Bruhat order.
\begin{definition}
  \label{def:bruhat-def}
  The \textit{Bruhat order} on the Coxeter group $W$ with generators
  $S$ is the transitive closure of the relation
  \[ u \to v \iff \ell(u) < \ell(v) \text{ and } u^{-1}v \in
  \bigcup_{w\in W} w
  \cdot S \cdot w^{-1}. \]
  So we say that $w \leq w'$ if there is a sequence of elements in $W$
  with
  \[ w \to w_1 \to w_2 \to \dots \to w_k \to w'. \]
\end{definition}

This ordering will show up in Theorem \ref{thm:bruhat-ord}. The
condition on $u^{-1}v$ is usually the one to check when we want to
find out if two elements are comparable, since if this is satisfied we
only need to check the lengths of $u$ and $v$ to find out if $u \to v$
or $v\to u$ holds. We will
show a quick example to see how the order works. Take the two
permutations $(1\;2\;3)$ and $(2\;3)$ in $S_3$. These are comparable,
since
\[ (2\;3)^{-1} \cdot (1\;2 \;3) = (1\;3) = (2\;3) (1\; 2) (2\; 3). \]
The minimum number of simple transpositions needed to write $(1\;2\;3)$
as a product is two, namely
\[ (1\;2\;3) = (1\;2) \cdot (2\;3), \]
while $(2\;3)$ is already written as a product with a single
factor. This shows
\[ \ell((2\;3)) = 1 < 2 = \ell((1\;2\;3)), \]
and we get the relation $(2\;3) \to (1\;2\;3)$, showing $(2\;3) \leq
(1\;2\;3)$ in the Bruhat order. On the other
hand, $(1\;2)$ and $(2\;3)$ are not comparable since they are both of
length one.

To illustrate the ordering, we can draw a diagram of $S_3$ where the
arrows denotes the relation $\sigma\to\tau$, and $\sigma\leq\tau$
exactly when we can follow a path from $\sigma$ to $\tau$ in the
diagram:
\[ \xymatrix{
  & (1\; 3) & \\
  (1\; 2\; 3)\ar[ur] & & (1\; 3\; 2)\ar[ul] \\
  (1\; 2)\ar[u]\ar[urr] & & (2\; 3) \ar[ull]\ar[u] \\
  & \Id \ar[ul] \ar[ur] &} \]

%%% Local Variables: 
%%% mode: latex
%%% TeX-master: "main"
%%% End: 
