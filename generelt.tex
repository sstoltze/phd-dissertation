% Her skal samles lidt generel information om SS og løkkerum

\chapter{Spectral sequences and loop spaces}
\fixme{Bedre titel...}

The purpose of this chapter is to give a brief introduction to two
important topics in this thesis, namely spectral sequences and loop
spaces. Spectral sequences are an important tool in algebraic topology
and can be considered a generalisation of long exact sequences, while
loop spaces are a class of spaces that appear naturally when
considering the spaces that will be studied in this thesis. The
following pages will collect some basic info and state the results
that are of interest in next chapters, while providing references for
proofs and further information. \fixme{Måske noget med at vi starter
  ca. fra Hatcher, alt andet betragtes som ``nyt''}

\section{Spectral sequences}
\label{sec:ss}

A spectral sequence, as we will be using them, is a tool for computing
the cohomology groups of a topological space, $X$, by considering an
increasing sequence of subspaces
\[ X_0 \subset X_1 \subset X_2 \subset \dots \subset X, \]
with the union covering all of $X$,
\[ X = \bigcup_{i} X_i. \]
An example that most people should be familiar with is the long exact
sequence of a pair, where
we only have a single subspace $A \subset X$. Another simple example
is the case where there are two subspaces,
\[ X = A \cup B, \]
in which case the cohomology of $X$ can often be computed from the
cohomology of $A$, $B$ and $A\cap B$ by using the Mayer-Vietoris
sequence. This generalises to a spectral sequence to compute
the cohomology of $X$ by working with an abitrary covering. As might
be expected, since spectral sequences work more generally they also
carry a corresponding increase in complexity when using them for
calculations. In the following section we will focus on the specific
case that 
we will be using in this text. For a general introduction to the
subject, see either \cite{hatcherss} or \cite{mccleary}.

\subsection{The spectral sequence of a filtration}
\label{sec:ss-filtration}

Consider a space $X$ with a filtration, that is an increasing sequence
of subspaces that cover $X$:
\[ X_0 \subset X_1 \subset X_2 \subset
\dots \subset X_k  = X. \]
All of the following can be modified to work if the filtration is not
finite, but we will not need this here. We will augment the filtration
by defining additional subspaces,
\[ X_s =
\begin{cases}
  \emptyset & s \leq -1 \\
  X & s \geq k+1
\end{cases}. \]

With such a sequence, we can
consider the relative cohomology groups
\[ E_1^{p,q} = H^{p+q}(X_p,X_{p-1}), \]
and the differentials between them coming from the long exact sequence
of the triple $(X_{p+1},X_p,X_{p-1})$:
\[ d_1 : E_1^{p,q} = H^{p+q}(X_p,X_{p-1}) \to H^{p+q+1}(X_{p+1},X_p) =
E_1^{p+1,q}. \]
This map factors as the composition of an
inclusion of the pair $(X_p,X_{p-1})$ and the boundary map of the pair
$(X_{p+1},X_p)$,
\[ \xymatrix{ d_1 : H^{p+q}(X_p,X_{p-1}) \ar[r]^<<<<<{\jmath^*} &
  H^{p+q}(X_p) \ar[r]^<<<<<{\delta} & H^{p+q+1}(X_{p+1},X_p),
} \]
and in particular we get that the composition of two such maps must be
zero since the composition $\delta \circ \jmath^* = 0$ by
exactness. So we can compute the homology of these groups and get new
groups,
\[ E_2^{p,q} = \ker d_1 / \im d_1. \]
More suprisingly, we can also find new maps
\[ d_2 : E_2^{p,q} \to E_2^{p+2,q-1} \]
that make these new groups into a chain complex, allowing us to form
\[ E_3^{p,q} = \ker d_2 / \im d_2. \]
This process can be continued, giving a sequence of groups and maps
\[ \xymatrix{d_r : E_r^{p,q} \ar[r] & E_r^{p+r,q-r+1} } \]
where the composition $d_r \circ d_r$ is zero and
$E_r^{p,q}$ is formed by taking the homology of $E_{r-1}^{p,q}$ with
the map $d_{r-1}$. Since we have $E_r^{p,q} = 0$ for $p < 0$ or $p >
k$, these maps will eventually always go either to or from a zero
group and hence the groups $E_r^{p,q}$ will not change when we
increase $r$. This allows us to
define the stable groups $E_\infty^{p,q}$ as these $r$-independent
groups. The following theorem relates them to the cohomology groups of
$X$ that we are interested in calculating.

\begin{theorem}
  \label{thm:ss}
  Given a filtration $\set{X_s}$ of a topological space $X$ as above,
  there is a sequence of abelian groups $E_r^{p,q}$ and maps
  \[ d_r : E_r^{p,q} \to E_r^{p+r,q-r+1} \]
  that satisfy the following:
  \begin{itemize}
  \item The first groups are
    \[ E_1^{p,q} = H^{p+q}(X_p,X_{p-1}) \]
    and the map $d_1$ is the boundary map of the triples
    $(X_{p+1},X_p,X_{p-1})$.
  \item The composition $d_r \circ d_r$ is zero and
    \[ E_{r+1}^{p,q} = \ker d_r / \im d_r, \]
    computed at $E_r^{p,q}$.
  \item There is a filtration of the $n\!$'th cohomology group of $X$,
    \[ 0 \subset F_n^n \subset \dots F_0^n = H^n(X), \]
    with the limit groups $E_\infty^{p,n-p}$ isomorphic to the
    quotients
    \[ E_\infty^{p,n-p} \cong F_p^n / F_{p+1}^n. \]
  \end{itemize}
\end{theorem}

This theorem is a combination of Proposition 1.2 and Theorem 1.14 in
\cite{hatcherss}, and the proof can be found there as well. Assuming
we can compute the spectral sequence, the group $H^n(X)$ can then be
found by solving extension problems of the form
\[ 0 \to F_{p+1}^n \to F_p^n \to E_\infty^{p,n-p} \to 0, \]
starting from the short exact sequence
\[ 0 \to E_\infty^{n,0} \to F_{n-1}^n \to E_\infty^{n-1,1} \to 0. \]
For example, if we have that the groups $E_{\infty}^{p,n-p}$ are free
for all $p$, the short exact sequences always split and $H^n(X)$ is
isomorphic to the direct sum of these groups,
\[ H^n(X) \cong \bigoplus_p E_\infty^{p,n-p}. \]

\section{Loop spaces}
\label{sec:ls}

Throughout this section, every topological space will have a
distinguished basepoint. We will be looking a spaces of loops,
consiting of maps from the circle to a given space $X$, preserving the
basepoint. For a general introduction to loop spaces, consult e.g.
\cite{may}. The results in this section are taken from
\cite{milnor} and the proofs can be found there as well.

\begin{definition}
  If $X$ is a topological space with basepoint $*\in
  X$, the loop space $\Omega(X)$ is the space of based maps from the
  cicle to $X$,
  \[ \Omega(X) = \set{ \g : (S^1,1)\to (X,*) } = \set{ \g : [0,1] \to X
    \mid f(0) = f(1) = * }. \]

  We equip this set with the compact-open topology, so for a compact
  set $K \subset S^1$ and an open set $U \subset X$, we get an open
  subset of the loop space consisting of functions mapping $K$ to $U$,
  \[ V(K,U) = \set{ \g\in \Omega(X) \mid \g(K) \subset U }. \]
  The collection of all such sets form a subbase for the compact-open
  topology on $\Omega(X)$.
\end{definition}

We will only worry about the case where $X$ is a Riemannian manifold,
so it comes equipped with a Riemannian metric $g$ and an induced
topological metric $d$. In this case, the compact-open topology is
induced by the metric
\[ d^*(\g,\g') = \max_t d(\g(t),\g'(t)). \]
By \cite[Theorem 17.1]{milnor}, this is homotopy equivalent to the
space of piecewise smooth loops in $X$, and we will generally work
with this space instead. The proof will not be repeated here, but
there is a construction in the proof that deserves a mention. For an
open cover $\set{X_\alpha}_{\a\in I}$ of $X$, we define subsets of the
loop space by requiring that the loops are piecewise contained in a
set in the cover,
\[ \Omega_k(X) = \set{\g \in \Omega(X) \delim \Big\vert\delim
  \text{For each } j \leq 2^k, \g_{\left|\left[
        \frac{j-1}{2^k},\frac{j}{2^k} \right]\right.} \text{ is 
    contained in $X_\alpha$ for some $\alpha$.} }. \]
This gives an increasing sequence of open subsets of $\Omega(X)$
that cover the entire space. In particular, a compact subset of
$\Omega(X)$ will be contained in $\Omega_k(X)$ for some $k$. By
defining a cover of $X$ with good properties, we will be able to work
exclusively with loops where we have some control
over their behaviour.


% Nævne at vi ikke behøver køre * -> *, men også kan køre p -> q

Note that instead of considering loops, we could also consider the
space of paths between two points $p,q\in X$, given by
\[ \Omega(X,p,q) = \set{ \g : [0,1] \to X \mid \g(0) = p, \g(1) = q
}. \]
But by composing such a path with a fixed path $\a$ from $q$ to $p$,
we end up with a loop at $p$. If we compose a loop at $p$ with the
path from $p$ to $q$ given by running $\a$ in reverse, we get a path
from $p$ to $q$, and the composition of these two operrations is
homotopic to the identity. This shows that all such spaces 
are homotopy equivalent, so we will not make a major distinction
between them.

\fixme{Et kort indblik i flag, ca. en halv eller en hel side?}

\section{Flags}
\label{sec:flags}

The last thing to be discussed is the concept of a flag in a finite
dimensional vector space. In an
$m$-dimensional vector space $V$, a flag is an increasing sequence of
subspaces of $V$,
\[ V_1 \subset V_2 \subset \dots \subset V_m = V, \]
with the dimension of $V_i$ equal to $i$. So the dimension of each
subspace is one greater than the last. Note that if we pick a non-zero
vector $v_1$ in $V_1$, another (non-zero) vector $v_2$ in $V_2 - V_1$,
and continue picking $v_i$ in $V_i - V_{i-1}$, we end up with $m$
linearly independent vectors. This allows us to construct a matrix in
$\GL(V)$ by considering
\[ [v_1,\dots,v_m], \]
and we can recover the flag from the matrix by considering
\begin{align*}
  V_1 =\, &\spa(v_1) \\
  V_2 =\, &\spa(v_1,v_2) \\
  \vdots\,\, & \\
  V_m =\, &\spa(v_1,\dots,v_m).
\end{align*}
Likewise, we could take the span of the columns starting from $v_m$
and working backwards to get another flag.
\begin{definition}
  Given an invertible matrix $A = [v_1,\dots,v_m]$, the \textit{left
    flag} of $A$ is
  \begin{alignat*}{5}
    \mathrm{Fl_L}(A)\, &&= \Big(&\spa(v_1) &&\subset \spa(v_1,v_2)
    &&\subset \dots &&\subset \spa(v_1,\dots,v_m) = V\Big).
  \end{alignat*}
  The \textit{right flag} of $V$ is defined similarly as
  \begin{alignat*}{5}
    \mathrm{Fl_R}(A)\, &&= \Big(&\spa(v_m) &&\subset
    \spa(x_{v-1},x_v) &&\subset \dots &&\subset
    \spa(v_1,\dots,v_m) = V\Big). 
  \end{alignat*}
\end{definition}
Note that we could multiply any column of $A$ by a non-zero scalar and
we would end up with the same flag. Likewise, if we replace $v_j$ by
$v_j + \lambda v_i$ for a scalar $\lambda$ and indices $j > i$, we do not
change the left flag of $A$. The right flag is unchanged by doing
the same operation with $j < i$. In particular, if $L$ is a
lower-triangular invertible matrix we can do the above operations one
column at a time, starting from the last, and see that
\[ \mathrm{Fl_R}(L) = \mathrm{Fl_R}(\Id). \]
And similarly, if $U$ is upper-triangular and invertible and $\sigma$
is a permutation matrix:
\[ \mathrm{Fl_L}(\sigma U) = \mathrm{Fl_L}(\sigma). \]
This is slightly harder to see, but $\sigma$ permutes the rows of $U$,
moving the first row to row $\sigma(1)$, the second row to row
$\sigma(2)$, and so on. By adding the first column to remaining ones,
with a suitable scalar multiplied on, we can ensure that the only
non-zero entry in row $\sigma(1)$ of $\sigma U$ is in the first
column, row $\sigma(1)$, just as it is in the matrix
$\sigma$. Continuing like this gives the desired identity.
\fixme{Mere her... Eller en afslutning}

\section{Coxeter groups and the Bruhat order}

\fixme{Skriv lidt her... Bare lige nok til at vise at permutationerne
  er relaterede}

%%% Local Variables: 
%%% mode: latex
%%% TeX-master: "main"
%%% End: 
