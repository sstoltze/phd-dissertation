
\chapter{Loops and limits} 
\label{chap:loekker}

We now return to the general case. We want to show that the limit
space
$X_{m,\infty}$ is equivalent to the space of loops in the flag
manifold $\GLB{m}$.

\begin{definition}
  The loop space $\Omega = \Omega(\GLB{m})$ is the space
  \[ \Omega = \set{\g : [0,1] \to \GLB{m} \mid \g(0) = \g(1) =
    \Id} \]
  with the compact-open topology. This was briefly described in
  Section \ref{sec:ls}. Since $\GLB{m}$ is a Riemannian manifold, the
  loop space becomes a metric space where loops are close if they are
  close at every point.
\end{definition}

\section{The map}

As has been done before, we will start by considering $Y_{m,n}$ and
then quotient out with the torus $\T^n$ afterwards.
For $A = (a_1,\dots,a_n)\in Y_{m,n}$ we
can define a path $ \g_A$ in $\GL_m(\C)$ by starting with
$\g_A(0) = \Id$. Since the matrix $(e_2,\dots,e_m,a_1)$ is
invertible by the definition of $Y_{m,n}$, we know that $a_1$ is
linearly independent of $e_2,\dots,e_m$. Hence it can be written as
\[ a_1 = \sum_{i=1}^m c_i e_i, \]
with $c_1 \neq 0$. Rewriting the expression leads to the formula
\[  \frac{1}{c_1} a_1= e_1 + \frac{1}{c_1} \left(\sum_{i=2}^m c_i
  e_i\right). \]
This gives us a path in $\GL_m(\C)$ from the identity to
$\left(\frac{1}{c_1} a_1,e_2,\dots,e_m\right)$:
\[ [0,1]\ni t \mapsto \left( e_1 + t\left(\sum_{i=2}^m
    \frac{c_i}{c_1}e_i\right),e_2,\dots,e_m \right) \in \GL_m(\C). \]
This process can be continued, since we can rewrite the similar
expression for $a_2$ as
\[ \frac{1}{d_2}a_2 = e_2 + \frac{1}{d_2}\left(\sum_{i=3}^m d_i
  e_i\right) + \frac{d_1}{d_2c_1}a_1, \]
and get a path
\[ t\mapsto \left(\frac{1}{c_1}a_1,e_2+t\left(\sum_{i=3}^m
    \frac{d_i}{d_2}e_i +
    \frac{d_1}{d_2c_1}a_1\right),e_3,\dots,e_m\right) \]
that starts where the other path stopped. If we continue through the
vectors in $A$, we will eventually end at a point,
\[ (\l_i e_i,\dots,\l_me_m,\l_1e_1,\dots,\l_{i-1}e_{i-1})\in \GL_m(\C), \]
which is diagonal, except the columns may have been
permuted if $n$ is not divisible by $m$. Choosing a fixed path in
$\GL_m(\C)$ from $(e_i,\dots,e_{i-1})$ to $\Id$, scaling the columns and
appending it to all the other paths then gives a path $ \g_A$ with
\[  \g_A(0) = \Id, \qquad \g_A(1) =
(\l_1e_1,\dots,\l_me_m). \]

If we consider this as a path in $\GLB{m}$ we
get a loop, since $\g_A(1)\in\B_m$ is a diagonal matrix. The
assignment
\[ f(A) = \g_A \in \Omega(\GLB{m}) \]
defines a continuous function from $Y_{m,n}$ to the loop space, as the
paths will be close at every point if the vectors we start from are
close. Note that
this function descends to the quotient space $X_{m,n}=Y_{m,n}/\T^n$, since
elements of $Y_{m,n}$ that are equivalent under the
$\mathrm{T}^n$-action gives
paths in $\GL_m(\C)$ that are equivalent under the $\mathrm{B}_m$-action.

The goal of this chapter is to prove the following theorem:

\begin{theorem}
  The map $f : X_{m,n} \to \Omega$ induces a map
  \[ f : X_{m,\infty} \to \Omega. \]
  This is a homotopy equivalence.
\end{theorem}

Since the space $\Omega$ is relatively well-studied, this gives
information about $X_{m,\infty}$. In particular, it allows us to
calculate the homology and cohomology of this space.

We want to show that the stabilisation map
\begin{align*}
  s : X_{m,n} &\to X_{m,n+1} \\
  A = (a_1,\dots,a_n) &\mapsto \left(La_1,\dots,La_n,
    \begin{pmatrix}
      1 \\
      \vdots\\
      1
    \end{pmatrix} 
\right) = (LA,Le_1) 
\end{align*}
does not affect the map $f$, or more precisely that $f$ induces a
well-defined map on $X_{m,\infty}$. To do this we will work with the
subsequence $X_{m,nm}$ of the sequence $X_{m,n}$, as the two limits
are isomorphic. This is a purely formal fact that follows from
stabilising $X_{m,r}$ to $X_{m,mn}$, which gives a map
\[ X_{m,\infty} \to  \varinjlim_{n} X_{m,mn}. \]
There is a similar map defined by considering $X_{m,mn}$ as the space
$X_{m,r}$ for $r = mn$, which also respects stablisation, and hence
defines a map
\[ \varinjlim_n X_{m,mn} \to X_{m,\infty}. \]
The composition of these two maps in either direction is the identity,
so the two limits are isomorphic.

We will show that the diagram
\[ \xymatrix{ X_{m,nm} \ar[rr]^{s^m}\ar[dr]_f & & X_{m,(n+1)m}
  \ar[dl]^f \\
  & \Omega & } \]
commutes up to homotopy. This is done by showing that the map $s^m$ is
homotopic to the map $\widetilde s$ that inserts the identity on the
last $m$ columns of $A \in X_{m,nm}$,
\[ \widetilde s (a_1,\dots,a_{nm}) =
(a_1,\dots,a_{nm},e_1,\dots,e_m). \]
If we connect $\Id$ and $L^{-m}$ by a path $\g$ in the space of
invertible lower-triangular matrices, we get a homotopy
\[ s_t(A) = (\g(t)L^mA,
\g(t)L^me_1,\g(t)L^{m-1}e_1,\dots,\g(t)Le_1) \]
which goes from $s_0 = s^m$ to the map
\[ s_1 : A \mapsto (A,e_1,L^{-1}e_1,\dots,L^{-m+1}e_1). \]
The matrix $T = (e_1,L^{-1}e_1,\dots,L^{-m+1}e_1)$ is upper-triangular
with entries given by the formula:
\[ T_{ij} = (L^{-j+1}e_1)_i = (-1)^{i-1}\binom{j-1}{i-1} \quad
\text{when } i \leq j. \]
This is proven by considering negative powers of $L$ and showing the
identity:
\[ (L^{-j+1})_{ik} = (-1)^{i-k}\binom{j-1}{i-k} \quad \text{when } i
\geq k. \] 
This is automatically true when $j$ is one, so assume it is true for
a given $j$. We will then show it for $j+1$. We are considering the
entry
\[ (L^{-(j+1)+1})_{ik} = (L^{-j})_{ik} = (L^{-1}\cdot L^{-j+1})_{ik} =
\sum_{l=1}^m (L^{-1})_{il} \cdot (L^{-j+1})_{lk}. \]
But if we use that $L^{-1}$ has entries
\[ (L^{-1})_{il} =
\begin{cases}
  -1 & l = i-1, \\
  1 & l = i,  \\
  0 & \text{otherwise},
\end{cases} \]
this simplifies to
\begin{align*}
  (L^{-(j+1)+1})_{ik} &= (-1)\cdot(-1)^{i-1-k}\binom{j-1}{i-1-k}
                        + (-1)^{i-k}\binom{j-1}{i-k} \\
                      &= (-1)^{i-k}\cdot \left( \binom{j-1}{i-k-1} +
                        \binom{j-1}{i-k} \right)\\
                      &= (-1)^{i-k} \binom{j}{i-k} \\
                      &= (-1)^{i-k} \binom{(j+1)-1}{i-k}.
\end{align*}
Since we are considering only the first column, we get
\[ T_{ij} = (L^{-j+1}e_1)_i = (L^{-j+1})_{i1} =
(-1)^{i-1}\binom{j-1}{i-1}. \]
As $T$ is an upper-triangular matrix, it can be homotoped to
the identity without affecting the linear independence of the
columns, and the maps $s^m$ and $\widetilde s$ are therefore
homotopic. The composition
$f\circ \widetilde s$ is the same as $f$, except that the loop
$f\circ\widetilde s$ is stationary at the identity for a while before
it terminates. Putting it all together shows the following:
\begin{lemma}
  \label{lem:homotopycommute}
  The diagram
  \[ \xymatrix{ X_{m,nm} \ar[rr]^{s^m}\ar[dr]_{f} & & \ar[dl]^f
    X_{m,(n+1)m} \\
    & \Omega & } \]
  commutes up to homotopy and $f$ induces a map:
  \[ f : X_{m,\infty} \to \Omega. \]
\end{lemma}

\section{Homotopy equivalence}

To show that the map $f : X_{m,\infty} \to \Omega$
gives an isomorphism of homotopy groups, we will again follow
\cite{milnor}. For each element $U$ of $\SUT{m}$, we can find a 
geodesically convex open ball centered on $U$. Since $\SUT{m}$
is compact, there is some $\e > 0$ such that any ball of radius $\e$
or less is geodesically convex. By the same argument, we can also
choose $\e$ so small that for any two points $A= (a_1,\dots, a_m)$ and
$B=(b_1,\dots,b_m)$ in $\SUT{m}$, with the distance from $A$ to $B$
less
than $\e$, any $m$ subsequent vectors in the sequence
$(a_1,\dots,a_m,b_1,\dots,b_m)$ are linearly
independent. 
With this, we get an increasing sequence of open sets in $\Omega$,
defined by requiring the loops to be piecewise contained in a ball of
radius $\e$:
\[ \Omega_k = \set{\g \in \Omega \delim \Big\vert\delim
  \g_{\left|\left[ \frac{j-1}{2^k},\frac{j}{2^k} \right]\right.} \text{ is
  contained in an $\e$-ball $\mathrm{B}_\e$} }. \]
By the choice of $\e$ made above, we can define a map
\begin{align*}
  \varphi_k : \Omega_k &\to X_{m,(2^k-1)m} \\
  \g &\mapsto \left( \g\left(\frac{1}{2^k}\right),
    \g\left(\frac{2}{2^k}\right), \dots,
    \g\left(\frac{2^k-1}{2^k}\right) \right).
\end{align*}
The map is continuous since two paths being close in $\Omega_k$ means
that the maximum distance between them is small. But this translates
to the vectors of the image being close together. The maps $\varphi_k$
and $\varphi_{k+1}$ are related in the following fashion:

\begin{lemma}
  \label{lem:com}
  The following diagram commutes up to homotopy:
  \[ \xymatrix{\Omega_k \ar@{^(->}[r] \ar[d]^{\varphi_k} & \Omega_{k+1}
  \ar[d]^{\varphi_{k+1}} \\
  X_{m,(2^k-1)m} \ar[r]^-{s} & X_{m,(2^{k+1}-1)m}. } \]
\end{lemma}
\begin{proof}
  The proof is by direct computation.
  \begin{align*}
    \varphi_{k+1} &= \left( \g \mapsto \left(
        \g\left(\frac{1}{2^{k+1}}\right),
        \g\left(\frac{2}{2^{k+1}}\right), \dots,
        \g\left(\frac{2^{k+1}-1}{2^{k+1}}\right) \right)\right) \\
    &\simeq \left(\g \mapsto \left( \g\left(\frac{2}{2^{k+1}}\right),
      \g\left(\frac{2}{2^{k+1}}\right),
      \g\left(\frac{4}{2^{k+1}}\right), \dots,
      \g\left(\frac{2^{k+1}}{2^{k+1}}\right) \right)\right) \\
    &\simeq \left(\g\mapsto \left( \g\left(\frac{1}{2^{k}}\right),
      \g\left(\frac{2}{2^{k}}\right), \dots,
      \g\left(\frac{2^k-1}{2^k}\right), \Id,\dots,\Id \right) \right)\\
    &= s\circ \varphi_k.
  \end{align*}
  The first homotopy is given by sliding along the minimal geodesic
  between the points, which gives an allowed path by our choice of
  $\e$. The second homotopy is somewhat similar: If $(a_1,\dots,a_m)$
  and $(b_1,\dots,b_m)$ are two elements of $\SUT{m}$ and we have an
  element of $X_{m,n}$ of the form
  \[ \left( \dots, a_1,\dots,a_m,a_1,\dots,a_m,b_1,\dots,b_m,\dots
  \right), \]
  then we can change the middle columns one by one, as we did in
  the construction of $f$, and get a path in $X_{m,n}$ that changes
  the middle from $(a_1,\dots,a_m)$ to $(b_1,\dots,b_m)$:
  \begin{align*}
    \left(a_1,\dots,a_m,a_1,a_2,\dots,a_m,b_1,\dots,b_m \right)
    &\leadsto
    \left(a_1,\dots,a_m,b_1,a_2,\dots,a_m,b_1,\dots,b_m\right) \\ 
    &\leadsto
    \left(a_1,\dots,a_m,b_1,b_2,\dots,a_m,b_1,\dots,b_m\right) \\  
    &\leadsto
    \left(a_1,\dots,a_m,b_1,b_2,\dots,b_m,b_1,\dots,b_m\right).  
  \end{align*}
  This allows us to take all the repetitions
  $\left(\dots,\g(t),\g(t),\dots\right)$ and move them to the end as
  diagonal matrices, which results in the stabilisation map $s$.
\end{proof}

The above lemma shows that the maps $\varphi_k$ fit together to define
a map from $\Omega$ to $X_{m,\infty}$.
The idea is to use the maps $\varphi_k$ as homotopy inverses of the
loop space map $f$. We are now ready to prove the result.

\begin{theorem}
  \label{thm:loekker}
  The map $f : X_{m,\infty} \to \Omega$ is a homotopy
  equivalence.
\end{theorem}

\begin{proof}
  We will prove that $f$ is a weak homotopy equivalence, i.e. the map
  \[f_* : \pi_i(X_{m,\infty}) \to \pi_i(\Omega)\]
  is an isomorphism for all $i$. Since the
  loop space is a CW complex and $X_{m,\infty}$ is a limit of CW
  complexes, by respectively Corollary 17.2 and the appendix of
  \cite{milnor}, this proves the theorem.
  
  We will start with surjectivity. Let
  \[ g : W \to \Omega \]
  be a finite cell complex in $\Omega$. We want to show that we can
  construct a map $\widetilde g$ from $W$ to $X_{m,\infty}$ such that
  the composition $f\circ \widetilde g$ is homotopic to $g$.
  
  The image of $g$ is compact and hence contained in $\Omega_k$ for
  some $k$, since $\set{\Omega_k}_{k\in\N}$ is an open cover of
  $\Omega$. Consider the map 
  \[ \widetilde g = \varphi_k\circ g : W \to X_{m,(2^k-1)m}. \]
  Evaluating at a point $w\in W$ gives
  \[ \widetilde g(w) = \varphi_k(\g_w) = \left(
    \g_w\left(\frac{1}{2^k}\right),\dots,
    \g_w\left(\frac{2^k-1}{2^k}\right) \right). \]
  Evaluating $f$ on this gives a path connecting the points
  $\g_w\left(\frac{j}{2^k}\right)$. But by our choice of $\e$, any
  such path is homotopic to the path connecting these points with
  minimal geodesics, so $f\circ\widetilde g \simeq g$.

  To show injectivity, let
  \[ g : W \to X_{m,\infty}\]
  be a finite cell complex with $f\circ g$ null-homotopic. We want to
  show that $g$ is null-homotopic in $X_{m,\infty}$. In the
  direct limit, a compact set is contained in the image of $X_{m,n}$
  for some $n$ (see \cite[Proposition~A.1]{hatcher} or
  \cite[Chapter~9.4]{may}), so $g$ is represented by a map:
  \[ g : W \to X_{m,n}. \]
  We will work with this representative. The map $f\circ
  g$ is null-homotopic in $\Omega$ and the image of the homotopy is
  compact, so it is
  contained in $\Omega_k$ for some $k$. By applying $\varphi_k$, we
  get a function
  \[ \varphi_k\circ f\circ g : W \to X_{m,(2^k-1)m}. \]
  Since $f\circ g$ is null-homotopic, this function is
  null-homotopic. But by the same argument that was used in
  Lemma \ref{lem:com}, the map
  \[ \varphi_k(f(g(w))) = \left(
    \g_w\left(\frac{1}{2^k}\right),\dots,\g_w\left(\frac{2^k-1}{2^k}
    \right)\right) \] 
  is homotopic to the stabilisation map
  \[ s(g(w)) = \left[g(w),\Id,\dots,\Id\right], \]
  with the homotopy given by sliding along the path $\g_w=f(g(w))$.
  So the map $g$ can be stabilised to a null-homotopic map, so it must
  be null-homotopic in $X_{m,\infty}$. This shows that $f_*$ is
  injective on homotopy groups.
\end{proof}

Since $f$ is a homotopy equivalence, it follows that it induces
an isomorphism on homology and cohomology groups. The Pontrjagin
homology ring of the loop space, with product given by loop
concatenation, has been calculated in \cite[Theorem~4.1]{grbic} as
\begin{align*}
  H_*(\Omega(\SUT{m})) \cong T(x_1,\dots,x_{m-1}) \otimes
  \Z[y_1,\dots,y_{m-1}]/I,
\end{align*}
where $T(x_1,\dots,x_{m-1})$ is the tensor algebra generated by
$x_1,\dots,x_{m-1}$. The product has the relation $x_k^2 =
x_px_q+x_qx_p = 2y_1$ for $1\leq k,p,q\leq m-1$ and $p\neq q$. The
degree of $x_i$ is 1 and the degree of $y_i$
is $2i$. By the above result, this is also the homology of
$X_{m,\infty}$. The groups are finitely generated and torsion free, so
by the Universal Coefficient Theorem the homology and cohomology
groups are isomorphic as abelian groups. Note that the first
cohomology group, 
\[ H^1(\SUT{m}) \cong \Z^{m-1}, \] 
is the same as the first cohomology of $X_{m,n}$, as
computed in Corollary \ref{cor:1-kohom}.

\section{Other coefficients}
\label{sec:coef}

In the above, we did not use the structure of the complex numbers in
any noticeable way. In particular, we could replace them by the reals
or the quaternions without any change to the proofs. This immediately
gives us two theorems, very similar to Theorem
\ref{thm:loekker}. These require computing Borel subgroups and maximal
tori for various spaces, an overview of which can be found in
e.g. \cite[Example 6.7]{malle}.

\begin{theorem}
  The space $X_{m,\infty}^\R$ is homotopy equivalent to the loop space
  of the real flag manifold,
  $\Omega(\GL_m(\R)/\mathrm{B}_m) = \Omega(\SO_m/\mathrm{T}^{k})$, where
  $\SO_m$ are the $m \times m$ special orthogonal matrices and
  $\mathrm{T^k}$ is a maximal torus in the special orthogonal group.
  
  By \cite{grbic}, the Pontrjagin homology ring of this space depends
  on whether $m$ is odd or even. For odd $m$, it is
  \[ H_*(\Omega(2k+1)/\T^k) \cong T(x_1,\dots,x_k)\otimes
  \Z[y_1,\dots,y_{k-1},2y_n,\dots,2y_{2k-1}] / I, \]
  where $x_i$ is degree 1, the degree of $y_i$ and $2y_i$ is
  $2i$, and $I$ is generated by the relations
  \begin{align*}
    x_1^2-y_1, x_i^2-x_{i+1}^2&\quad 1 \leq i \leq k-1, \\
    x_a x_b + x_b x_a&\quad a \neq b, \\
    y_i^2-2y_{i-1}y_{i+1}+\dots\pm 2y_{2i}&\quad 1 \leq i \leq k-1.
  \end{align*}

  For even $m$, the homology ring is
  \[ H_*(\Omega(2k)/\T^k) \cong T(x_1,\dots,x_k)\otimes
  \Z[y_1,\dots,y_{k-2},y_{k-1}+z,y_{k-1}-z,2y_k\dots,2y_{2k-2}] /
  I, \]
  where $x_i$ is degree 1 and anything involving a $y_i$ is of degree
  $2i$. The relations are generated by
  \begin{align*}
    x_1^2-y_1, x_i^2-x_{i+1}^2& \quad 1 \leq i \leq k-1, \\
    x_ax_b+x_bx_a& \quad a\neq b, \\
    y_i^2-2y_{i-1}y_{i+1}+\dots\pm 2y_{2i}&\quad 1 \leq i \leq k-2,
    \\
    (y_{n-1}+z)(y_{n-1}-z)-2y_{n-2}y_n+\dots\pm 2y_{2n-2}&. 
  \end{align*}
\end{theorem}

\begin{theorem}
  The space $X_{m,\infty}^\Hq$ is homotopy equivalent to the loop
  space of the quaternionic flag manifold
  $\Omega(\GL_m(\Hq)/\mathrm{B}_m) = \Omega(\mathrm{Sp}_m/\T^m)$.
  Here, $\mathrm{Sp}_m$ is the symplectic group and $\T^m$ is a
  maximal torus.

  The homology ring is
  \[ H_*(\Omega(\mathrm{Sp}_m/\T^m)) \cong T(x_1,\dots,x_m)\otimes
  \Z[y_2,\dots y_m] / I, \]
  with $x_i$ of degree 1 and $y_i$ of degree $4i-2$. The relations are
  \begin{align*}
    x_i^2 = x_j^2&,\\
    x_a x_b = -x_b x_a& \quad a\neq b.
  \end{align*}
\end{theorem}

This gives the homology of $X_{m,\infty}^\R$ and
$X_{m,\infty}^\Hq$.


\section{Other stabilisations}
\label{sec:stabilitet}

Now that we know the space $X_{m,\infty}$, we can consider different
ways of reaching this space from $X_{m,n}(\sigma)$. The
first thing to note is that the space $X_{m,\infty}$, so far described
as sequences of linearly independent vectors $(v_1,v_2,\dots,v_k)$ in
$\C^m$ of varying length, may as well be described as infinite
sequences $(v_i)_{i=1}^\infty$ for which there is some $k$ such that
the matrix $(v_{k+rm +1},v_{k+rm +2},\dots,v_{k+(r+1)m})$ is the
identity matrix for all integers $r \geq 0$, with the same
requirements for
linear independence. We could also consider sequences that are
infinite in the other direction, or
doubly infinite sequences
that start and end with the identity matrix,
and get the same space. Likewise, $X_{m,\infty}(\sigma)$ could be
defined as doubly infinite series that start with the identity matrix
and end with infinite copies of $\sigma$. The homeomorphisms between
these spaces are the obvious ones of inclusion or cutting off the tail
or start of an infinite sequence. As we saw in the proof of Lemma
\ref{lem:homotopycommute}, the two inclusions of $X_{m,n}$ in
$X_{m,\infty}$ by either
using the stabilisation map $s$ or by adding on the identity matrix
are homotopic, so this does not change any of the above results. To
differentiate the two ways of stabilising, we will refer to the
inclusion by adjoining an
infinite sequence of identity matrices on the left as
\[ L_n : X_{m,n}(\sigma) \to X_{m,\infty}(\sigma). \]

This map is important since it respects the
filtration defined in Section \ref{sec:filtration},
\[ F_0 \subset F_1 \subset \dots \subset X_{m,n}(\sigma), \]
and hence defines a map between the spectral sequences of
$X_{m,n}(\sigma)$ and $X_{m,\infty}(\sigma)$. These spectral sequences
have not played a role so far for $\sigma\neq \Id$, but are defined
exactly as for $X_{m,n}$.

We will be using $L_n$ to prove the following theorem.
\begin{theorem}
  \label{thm:kollaps}
  The spectral sequence for $X_{3,\infty}(\sigma)$ collapses on the
  second page, with all cohomology concentrated in the first column.
\end{theorem}

At first this theorem may seem slightly strange, but we know
that the spaces $X_{3,\infty}(\sigma)$ are homotopy equivalent for all
$\sigma$, since they are all equivalent to the loop space
$\Omega(\SU_3/\T^2)$, so the theorem
just tells us that all the cohomology of $X_{3,\infty}(\sigma)$ can be
found in the $(X_{3,\infty})_{\emptyset}$-part of $F_0 \cong
(X_{3,\infty})_{\emptyset}\times T^2$.

\begin{proof}
  We will show that the spectral sequence for $X_{3,\infty}(\sigma)$
  is determined entirely by the torus-part. To see this, note that
  since all the spaces $X_{3,\infty}(\sigma)$ are equivalent, they all
  have the same cohomology ring $R$. The first page of the spectral
  sequence has the form
  \[ R \otimes H^*(\T^2) \to (R\oplus R)\otimes H^*(T\times (D,D-0))
  \to R\otimes H^*(D_2,D_2-0). \]
  The boundary map $d_1$ is an inclusion of spaces
  followed by the boundary map for the pair $(D,D-0)$,
  \[ d_1 = \delta \circ \imath^*, \]
  and we would like to show that the inclusion map
  essentially does nothing. We know that for cohomology classes $r$
  and $t$, we have
  \[ \imath^*(r \otimes t) = \imath^*((r\otimes 1)\cdot (1\otimes t))
  = \imath^*(r\otimes 1) \imath^*(1\otimes t), \]
  so we need to know that happens in both these cases. Instead of
  reducing all the way to what would correspond to $X_{3,n-1}(\tau)$,
  as we have often done in the finite-dimensional case, it is in this
  case simpler to only use
  the torus action to change all of
  the non-zero entries in the last column to one and avoid the
  multiplication with the lower-triangular matrix. The inclusion
  can be illustrated by considering a concrete example. For the
  case
  \[\imath : X_{3,\infty}(\Id)_{\set{1}}\times(D-0) \to
  X_{3,\infty}(\Id)_{\emptyset}, \]
  we get the map
  \[ \left(\begin{pmatrix}
      \matrow{r_1} & 0   \\
      \matrow{r_2} & t_2 \\
      \matrow{r_3} & 1
    \end{pmatrix},\varepsilon\right) =
  \left(\begin{pmatrix}
      \matrow{r_1} & 0 \\
      \matrow{r_2} & 1 \\
      \matrow{r_3} & 1
    \end{pmatrix}, t_2, \varepsilon\right) \mapsto
  \left(\begin{pmatrix}
      \matrow{\frac{1}{\varepsilon}r_1} & 1 \\
      \matrow{r_2} & 1 \\
      \matrow{r_3} & 1
    \end{pmatrix}, t_2,\varepsilon\right). \]
  
  When $\varepsilon$ is a fixed, small real number, we get a homotopy
  commutative diagram
  \[ \xymatrix{ X_{3,\infty}(\Id)_{\set{1}} \ar[d]^{\imath}
    \ar@{^(->}[r] & X_{3,\infty}(\Id) \\
    X_{3,\infty}(\Id)_{\emptyset} \ar@{^(->}[ur] }, \]
  where the homotopy is given by first multiplying the top row with a
  number moving from one to $\varepsilon$ and then moving the last
  coordinate of the top row to zero, illustrated below:
  \begin{align*}
    \begin{pmatrix}
      \matrow{\frac{1}{\varepsilon}r_1} & 1 \\
      \matrow{r_2} & 1 \\
      \matrow{r_3} & 1
    \end{pmatrix} 
                   &\leadsto 
                     \begin{pmatrix}
                       \matrow{r_1} & \varepsilon \\
                       \matrow{r_2} & 1 \\
                       \matrow{r_3} & 1
                     \end{pmatrix} \\
                                        &\leadsto
                                          \begin{pmatrix}
                                            \matrow{r_1} & 0 \\
                                            \matrow{r_2} & 1 \\
                                            \matrow{r_3} & 1
                                          \end{pmatrix}.
  \end{align*}
  But the inclusions into $X_{3,\infty}(\Id)$ are homotopy
  equivalences, since they correspond to considering paths with a
  different endpoint in the loop formulation and we know these are all
  equivalent. Hence we must have $\imath$ a homotopy equivalence as
  well, and so it is an isomorphism on cohomology groups:
  \[ \imath^*(r\otimes 1) = r\otimes 1. \]
  
  Likewise, we can see that if we choose our matrix to be e.g.
  \[ A = \begin{pmatrix}
    1 & 0 & 0 \\
    0 & 1 & 1 \\
    0 & 0 & 1
  \end{pmatrix}, \]
  the inclusion does not change the rows of $A$ since we have
  quotiented out the torus action on the columns. Hence we conclude
  that we preserve the other part of the cohomology as well. All in
  all, we get
  \[ d_1(r\otimes t) = r \otimes \delta(t). \]
  But we know the map $\delta$, and hence we can compute the map on
  the first
  column as
  \begin{align*}
    d_1(r \otimes t_1t_2) &= ( r\otimes t_2, - r\otimes t_1) \otimes
    [D,D-0], \\
    d_1(r \otimes t_1) &= (0, -r\otimes t_1) \otimes [D,D-0], \\
    d_1(r \otimes t_2) &= (r\otimes t_2,0) \otimes [D,D-0], \\
    d_1(r \otimes 1) &= 0.
  \end{align*}

  By doing the exact same thing for the second column, leaving out
  the class $[D,D-0]$ to ease on notation, we see
  \begin{align*}
    d_1( r_1\otimes t_2, r_2\otimes t_1) &= (r_1 + r_2) \otimes
                                           [D_2,D_2-0], \\
    d_1( r_1 \otimes t_2, 0) &= 0, \\
    d_1( 0, r_2\otimes t_1) &= 0.
  \end{align*} 
  But by using this we can compute the second page and see that it is
  concentrated as $R\otimes 1$ in the first column, as desired.
\end{proof}


\section{Cohomological stability}
\label{sec:costa}

Since we now know the space $X_{m,\infty}(\sigma)$ and its cohomology, we
would like to relate it to the cohomology of
$X_{m,n}(\sigma)$, for finite $n$. We once again specialise to the
case $m = 3$ in
everything that
follows. First, we need an offset that depends on the permutation
we are considering.
Define a function $\lambda$ on the symmetric group of three elements
by
\begin{align*}
  \lambda(\sigma) &=
                    \begin{cases}
                      1 & \sigma = \Id, \\
                      2 & \sigma \in \set{(1\,2),(2\,3),(1\,2\,3)},
                      \\
                      3 & \sigma \in \set{(1\,3\,2), (1\,3)}.
                    \end{cases}
\end{align*}

We will need to distinguish the spectral sequences for various
permutations. We will refer to the groups of the spectral sequence
used to compute the cohomology of $X_{3,n}(\sigma)$ as
$E_r^{p,q}(n,\sigma)$. We will also use the notation
$\sigma_{\emptyset}$ for the permutation that shows up in the
column $p=0$ when using the spectral sequence to compute cohomology,
corresponding to
\[X_{3,n}(\sigma)_{\emptyset} \cong
X_{3,n-1}(\sigma_{\emptyset})\times \T^2,\]
and likewise for $\sigma_{\set{1}},
\sigma_{\set{2}}$ and $\sigma_{\set{1,2}}$.

This section will prove the following theorem:
\begin{theorem}
  \label{thm:kostab}
  If $n \geq k + \lambda(\sigma)$, the stabilisation map
  \[ L_{n}^* : H^k(X_{3,\infty}(\sigma)) \to H^k(X_{3,n}(\sigma)) \]
  is an isomorphism.
\end{theorem}

\begin{proof}
  The proof will proceed by induction on $k$. Our induction hypothesis
  is divided into two parts, we require the stabilisation to be an
  isomorphism when $n$ is greater than $k+\lambda(\sigma)$ and
  injective when $n$ is $k+\lambda(\sigma)-1$. The exact statement we
  want to prove is the following:
  \begin{hypothesis}[$I_{k-1}$]
    For all $r \leq k-1$ and all $\sigma \in S_3$, the following two
    statements hold:
    \begin{itemize}
    \item The map $L_{n}^* : H^r(X_{3,\infty}(\sigma)) \to
      H^r(X_{3,n}(\sigma))$ is an isomorphism for $n \geq r +
      \lambda(\sigma)$.
    \item The map $L_{n}^* : H^r(X_{3,\infty}(\sigma)) \to
      H^r(X_{3,n}(\sigma))$ is injective for $n \geq r +\lambda(\sigma)
      -1$.
    \end{itemize}
  \end{hypothesis}
  
  By direct calculation, the
  groups $H^0(X_{3,n}(\sigma))$ are all $\Z$ when $n \geq
  \lambda(\sigma)-1$, so the isomorphism part of $I_0$ is
  automatically true while the injective map also becomes an
  isomorphism.
  
  We now assume the hypothesis $I_{k-1}$ and want to prove $I_k$. Before
  we start, we consider the spectral sequence for $X_{3,n+1}(\sigma)$
  and observe how the different permutations that show up relate to each
  other. This is summarised in the following table:
  
  \[
  \begin{array}{|c|cccc|}
    \hline
    \sigma & \multicolumn{1}{c|}{\sigma_\emptyset} 
    & \multicolumn{1}{c|}{\sigma_{\set{1}}} 
    & \multicolumn{1}{c|}{\sigma_{\set{2}}} & \sigma_{\set{1,2}} \\
    \hline
    \Id & \Id & (1\;2) & (2\;3) & (1\;3\;2) \\
    (1\;2) & \Id & (1\;2\;3) & \Id & (1\;3) \\
    (2\;3) & \Id & (1\;2) & \Id & (1\;2) \\
    (1\;2\;3) & \Id & \Id & \Id & \Id \\
    (1\;3\;2) & (2\;3) & (1\;2\;3) & (2\;3) & (1\;2\;3) \\
    (1\;3) & (2\; 3) & (2\;3) & (2\;3) & (2\;3) \\
    \hline
  \end{array}
  \]
  By considering the table, we can see the following inequalities for
  the function $\lambda$:
  \begin{align*}
    \lambda(\sigma) &\geq \lambda(\sigma_{\emptyset}), \\
    \lambda(\sigma) &\geq \lambda(\sigma_{\set{1}}) - 1, \\
    \lambda(\sigma) &\geq \lambda(\sigma_{\set{2}}) - 1, \\
    \lambda(\sigma) &\geq \lambda(\sigma_{\set{1,2}}) - 2. \\
  \end{align*}
  These will be used during the proof.
  
  
  
  \subsection{Surjectivity of $s^*$}
  
  Let $\sigma$ be a permutation in $S_3$ and let $n \geq k +
  \lambda(\sigma)$. We start by showing that the map
  \[ s^* : H^k(X_{3,n+1}(\sigma)) \to
  H^k(X_{3,n}(\sigma_{\emptyset})) \] 
  is surjective.
  To see this, we will look at the spectral sequences for
  $X_{3,\infty}(\sigma)$ and $X_{3,n+1}(\sigma)$, and the maps between
  them induced by the stabilisation map $L^*_{n+1}$. This is
  summarised in the following diagram.
  \[ \xymatrix{ H^k(X_{3,\infty}(\sigma))
    \ar@{^(->}[r]^-{\imath^*}\ar[d]^{L^*_{n+1}} & E_1^{0,k}(\infty,\sigma)
    \ar[r]^{d_1}\ar[d]^{A_0} & 
    E_1^{1,k}(\infty,\sigma) \ar[r]^{d_1}\ar[d]^{A_1} &
    E_1^{2,k}(\infty,\sigma) \ar[d]^{A_2} \\
     H^k(X_{3,n+1}(\sigma)) \ar@{^(->}[r]^-{\imath^*} &
    E_1^{0,k}(n+1,\sigma) \ar[r]^{d_1} & E_1^{1,k}(n+1,\sigma)
    \ar[r]^{d_1} & E_1^{2,k}(n+1,\sigma).
  } \]
  The upper row is exact, since it is part of the spectral sequence for
  $X_{3,\infty}(\sigma)$ and this collapses at the second page by
  Theorem \ref{thm:kollaps}. The
  bottom row is a chain complex, by considering the spectral sequence,
  and is exact at
  $E_1^{0,k}(n+1,\sigma)$. To see this, consider the induction
  hypothesis applied to the spectral sequence for $X_{3,n+1}(\sigma)$,
  illustrated in Figure \ref{fig:surjektiv}. Since we choose to
  stabilise from the left,
  as discussed in Section \ref{sec:stabilitet} above, $L_n$
  becomes a map of spectral sequences and respects the stabilisation map
  $s$. We use the notation \colorbox{isomorphism}{$E_1^{p,q}$} to show
  that the map $L_n^*$ is an isomorphism on $E_1^{p,q}$ according to
  $I_{k-1}$. 
  \begin{figure}[ht]
    \[ 
    \begin{array}{|c|ccc|}
      \hline 
      &
      &
      &
      \\
      \multirow{6}{*}{$k$} 
      & \colorbox{isomorphism}{$H^{k-2}(X_{3,n}(\sigma_{\emptyset}))
        \otimes t_1t_2 $}
      & \colorbox{isomorphism}{ $H^{k-2}(X_{3,n}(\sigma_{\set{1}}))
        \otimes t_2 $}
      & \multirow{6}{*}{\colorbox{isomorphism}{$
        H^{k-2}(X_{3,n}(\sigma_{\set{1,2}}))$}} \\
      & \colorbox{isomorphism}{$H^{k-1}(X_{3,n}(\sigma_{\emptyset}))
        \otimes t_1$}
      & \colorbox{isomorphism}{$H^{k-2}(X_{3,n}(\sigma_{\set{2}}))
        \otimes t_1$}
      & \\
      & \colorbox{isomorphism}{$H^{k-1}(X_{3,n}(\sigma_{\emptyset}))
        \otimes t_2 $} 
      & \colorbox{isomorphism}{$H^{k-1}(X_{3,n}(\sigma_{\set{1}}))
        \otimes 1$} &\\
      & H^k(X_{3,n}(\sigma_{\emptyset})) \otimes 1 
      & \colorbox{isomorphism}{$H^{k-1}(X_{3,n}(\sigma_{\set{2}}))
        \otimes 1$} &\\
      & & &\\
      \hline& & & \\
      \multirow{6}{*}{$k-1$}
      & \colorbox{isomorphism}{$H^{k-3}(X_{3,n}(\sigma_{\emptyset}))
        \otimes t_1t_2$}
      & \colorbox{isomorphism}{$H^{k-3}(X_{3,n}(\sigma_{\set{1}}))
        \otimes t_2$}
      & \multirow{6}{*}{\colorbox{isomorphism}{$
        H^{k-2}(X_{3,n}(\sigma_{\set{1,2}}))$}} \\
      & \colorbox{isomorphism}{$H^{k-2}(X_{3,n}(\sigma_{\emptyset}))
        \otimes t_1$}
      & \colorbox{isomorphism}{$H^{k-3}(X_{3,n}(\sigma_{\set{2}}))
        \otimes t_1$} &\\
      & \colorbox{isomorphism}{$H^{k-2}(X_{3,n}(\sigma_{\emptyset}))
        \otimes t_2 $}
      & \colorbox{isomorphism}{$H^{k-2}(X_{3,n}(\sigma_{\set{1}}))
        \otimes 1$}
      &\\
      & \colorbox{isomorphism}{$H^{k-1}(X_{3,n}(\sigma_{\emptyset}))
        \otimes 1$}
      & \colorbox{isomorphism}{$H^{k-2}(X_{3,n}(\sigma_{\set{2}}))
        \otimes 1$}
      &\\
      && & \\
      \hline
      & & & \\
      & \multirow{4}{*}{\dots} 
      & \multirow{4}{*}{\dots} 
      & \multirow{4}{*}{\dots} 
      \\
      & & & \\
      & & & \\
      & & & \\
      & & & \\
      \hline && & \\
      \multirow{6}{*}{$0$} 
      & \multirow{6}{*}{\colorbox{isomorphism}{$E_{1}^{0,0}$}} 
      & \multirow{6}{*}{\colorbox{isomorphism}{$E_1^{1,0}$}}
      & \multirow{6}{*}{\colorbox{isomorphism}{$E_1^{2,0}$}} \\
      & & &\\
      & & &\\
      & & &\\
      & & &\\
      \hline
    \end{array}
    \]
    \caption{The $E_1$ page when we assume $I_{k-1}$ and $n \geq
      k+\lambda(\sigma)$.}
    \label{fig:surjektiv}
  \end{figure}
  
  If we pass to the second page of the spectral sequence, we
  can see that at row $k$ the groups will no longer change. This is
  because everything except $H^k(X_{3,n}(\sigma_{\emptyset}))\otimes 1$ is
  isomorphic to the $X_{3,\infty}(\sigma)$-case, where the columns $p=1$
  and $p=2$ disappear on the second page, according to Theorem
  \ref{thm:kollaps}. We then know that
  \[ H^k(X_{3,n+1}(\sigma)) = \ker d_1, \]
  and the inclusion into $E_1^{0,k}(n+1,\sigma)$ is injective.
  The map $A_1$, defined in the diagram, is an isomorphism, since it
  is induced from the maps
  \[ L^*_{n} : H^{k-1}(X_{3,\infty}(\sigma_{\set{i}})) \to
  H^{k-1}(X_{3,n}(\sigma_{\set{i}})). \]
  These are isomorphisms by applying the first part of $I_{k-1}$,
  using the assumption 
  \[ n \geq k+\lambda(\sigma) \geq
  (k-1)+\lambda(\sigma_{\set{i}}). \]
  The map $A_2$ is an isomorphism by the exact same argument.
  This allows us to do a diagram chase: Consider an element $x$ in
  $E_1^{0,k}(n+1,\sigma)$. Then $d_1(x) = A_1(y)$ for a $y$ in
  $E_1^{1,k}(\infty,\sigma)$. But 
  \[ A_2(d_1(y)) = d_1(A_1(y)) = d_1(d_1(x)) = 0, \]
  so $y = d_1(z)$ for some $z$ in $E_1^{0,k}(\infty,\sigma)$ by
  exactness. The element $x - A_0(z)$ is in the kernel of $d_1$,
  \[ d_1(x - A_0(z)) = d_1(x) - A_1(d_1(z)) = d_1(x) - A_1(y) = 0. \]
  So there is a $w$ in $H^k(X_{3,n+1}(\sigma))$ with $\imath^*(w)$
  equal to $x-A_0(z)$. So we have found $w$ and $z$ with
  \[ A_0(z) + \imath^*(w) = A_0(z) + x - A_0(z) = x, \]
  showing that
  \[ \xymatrix{ H^k(X_{3,n+1}(\sigma)) \oplus E_1^{0,k}(\infty,\sigma)
    \ar[r]^-{\imath^* + A_0} & E_1^{0,k}(n+1,\sigma) } \]
  is surjective. The map $s^*$ that we want to show is surjective is
  the composition
  \[ \xymatrix{ H^k(X_{3,n+1}(\sigma)) \ar[r]^-{\imath^*} &
    E_1^{0,k}(n+1,\sigma) = H^k(X_{3,n}(\sigma_{\emptyset})\times T^2)
    \ar[r]^-{\pr} & H^k(X_{3,n}(\sigma_{\emptyset})). } \]
  The projection, which is induced from the inclusion of
  $X_{3,n}(\sigma_{\emptyset})$ into the product, is surjective, so we
  only need to show that the image of
  $E_1^{0,k}(\infty,\sigma)$ under the map $\pr\circ A_0$ is contained
  in the image of $s^*$ to
  finish the proof of surjectivity of $s^*$. Consider the commutative
  diagram
  \[ \xymatrix{ E_1^{0,k}(\infty,\sigma) \ar[r]^{A_0} \ar[d]^{\pr}
    & E_1^{0,k}(n+1,\sigma) \ar[d]^{\pr} \\
    H^k(X_{3,\infty}(\sigma_{\emptyset})) \ar[r]^{L_n^*} &
    H^k(X_{3,n}(\sigma_{\emptyset})). } \]
  This shows that $\pr\circ A_0$ is contained in the image of
  $L_n^*$. But the stabilisation map $L_n$ factors up to homotopy over
  $s$, $L_n \simeq L_{n+1}\circ s$, so we get that
  \[ \im(\pr\circ A_0) \subset \im L_n^* \subset \im s^*. \]
  By applying the surjectivity of $\pr\circ (\imath^*+A_0)$ that was
  just shown, we see that
  \[  s^* : H^k(X_{3,n+1}(\sigma)) \to
  H^k(X_{3,n}(\sigma_{\emptyset})) \] 
  is surjective.
  
  \subsection{Injectivity}
  
  We will now prove injectivity of $L_n^*$, so assume that $n \geq
  k+\lambda(\sigma)-1$. As above, we will prove that
  the stabilisation map
  \[ s^* : H^k(X_{3,n+1}(\sigma)) \to
  H^k(X_{3,n}(\sigma_{\emptyset})) \] 
  is injective. Consider the spectral sequence for computing
  $H^k(X_{3,n+1}(\sigma))$. Figure \ref{fig:injektiv} is an image of
  the first page of the spectral sequence, with
  \colorbox{isomorphism}{$E_1^{p,q}$} denoting that $L_n^*$ is an
  isomorphism on the group and \colorbox{injective}{$E_1^{p,q}$}
  denoting that it is injective.
  \begin{figure}[ht]
    \[ 
    \begin{array}{|c|ccc|}
      \hline 
      & & & \\
      \multirow{6}{*}{$k$} 
      & \colorbox{isomorphism}{$H^{k-2}(X_{3,n}(\sigma_{\emptyset}))
        \otimes t_1t_2 $}
        & \colorbox{isomorphism}{ $H^{k-2}(X_{3,n}(\sigma_{\set{1}}))
          \otimes t_2 $}
          & \multirow{6}{*}{\colorbox{injective}{$
            H^{k-2}(X_{3,n}(\sigma_{\set{1,2}}))$}} \\
      & \colorbox{isomorphism}{$H^{k-1}(X_{3,n}(\sigma_{\emptyset}))
        \otimes t_1$}
        & \colorbox{isomorphism}{$H^{k-2}(X_{3,n}(\sigma_{\set{2}}))
          \otimes t_1$}
          & \\
      & \colorbox{isomorphism}{$H^{k-1}(X_{3,n}(\sigma_{\emptyset}))
        \otimes t_2 $} 
        & \colorbox{injective}{$H^{k-1}(X_{3,n}(\sigma_{\set{1}}))
          \otimes 1$} &\\
      & H^k(X_{3,n}(\sigma_{\emptyset})) \otimes 1 
        & \colorbox{injective}{$H^{k-1}(X_{3,n}(\sigma_{\set{2}}))
          \otimes 1$} &\\
      & & &\\
      \hline& & & \\
      \multirow{6}{*}{$k-1$}
      & \colorbox{isomorphism}{$H^{k-3}(X_{3,n}(\sigma_{\emptyset}))
        \otimes t_1t_2$}
        & \colorbox{isomorphism}{$H^{k-3}(X_{3,n}(\sigma_{\set{1}}))
          \otimes t_2$}
          & \multirow{6}{*}{\colorbox{isomorphism}{$
            H^{k-2}(X_{3,n}(\sigma_{\set{1,2}}))$}} \\
      & \colorbox{isomorphism}{$H^{k-2}(X_{3,n}(\sigma_{\emptyset}))
        \otimes t_1$}
        & \colorbox{isomorphism}{$H^{k-3}(X_{3,n}(\sigma_{\set{2}}))
          \otimes t_1$} &\\
      & \colorbox{isomorphism}{$H^{k-2}(X_{3,n}(\sigma_{\emptyset}))
        \otimes t_2 $}
        & \colorbox{isomorphism}{$H^{k-2}(X_{3,n}(\sigma_{\set{1}}))
          \otimes 1$}
          &\\
      & \colorbox{isomorphism}{$H^{k-1}(X_{3,n}(\sigma_{\emptyset}))
        \otimes 1$}
        & \colorbox{isomorphism}{$H^{k-2}(X_{3,n}(\sigma_{\set{2}}))
          \otimes 1$}
          &\\
      && & \\
      \hline
      & & & \\
      & \multirow{4}{*}{\dots} 
        & \multirow{4}{*}{\dots} 
          & \multirow{4}{*}{\dots} 
      \\
      & & & \\
      & & & \\
      & & & \\
      & & & \\
      \hline && & \\
      \multirow{6}{*}{$0$} 
      & \multirow{6}{*}{\colorbox{isomorphism}{$E_{1}^{0,0}$}} 
        & \multirow{6}{*}{\colorbox{isomorphism}{$E_1^{1,0}$}}
          & \multirow{6}{*}{\colorbox{isomorphism}{$E_1^{2,0}$}} \\
      & & &\\
      & & &\\
      & & &\\
      & & &\\
      \hline
    \end{array}
    \]
    \caption{The $E_1$ page when we assume $I_{k-1}$ and $n \geq
      k+\lambda(\sigma)-1$.}
    \label{fig:injektiv}
  \end{figure}
  
  If the map 
  \[ s^* : H^k(X_{3,n+1}(\sigma)) \to
  H^k(X_{3,n}(\sigma_{\emptyset})) \]
  is not injective, there is an element $\alpha$ of the kernel of the
  boundary map that is also in the group 
  \[ H^{k-1}(X_{3,n}(\sigma_{\emptyset})) \otimes t_1 \oplus 
  H^{k-1}(X_{3,n}(\sigma_{\emptyset})) \otimes t_2 \oplus 
  H^{k-2}(X_{3,n}(\sigma_{\emptyset})) \otimes t_1 t_2. \]
  But on these last groups, the stabilisation map is an isomorphism,
  so we get an element $\widehat{\alpha}$ in the corresponding group
  in the spectral sequence for $X_{3,\infty}(\sigma)$. Since the
  stabilisation maps are maps of spectral sequences, we have
  \[ 0 = d_1(L_{n}(\widehat{\alpha})) = L_n(d_1(\widehat{\alpha})). \]
  But the right hand side is the composition of $L_n$, which we know by
  the induction hypothesis to be injective, and the boundary map in the
  spectral sequence for $X_{3,\infty}(\sigma)$, which we know to be
  injective on the groups we are considering by Theorem
  \ref{thm:kollaps}. So $\widehat{\alpha}$ must
  be zero, and hence so is $\alpha$. This shows that $s^*$ is injective
  when $n \geq k + \lambda(\sigma) - 1$. Now we consider the different
  choices of $\sigma$.
  
  \subsubsection{$\sigma = \Id$}
  
  Since $\sigma_{\emptyset}$ is $\Id$, and all the maps
  \[ s^* : H^k(X_{3,j+1}) \to H^k(X_{3,j}), \quad j \geq n, \]
  are surjective, we know that the cohomology
  of $X_{3,\infty}$ is the inverse
  limit of the cohomology groups of $X_{3,n}$ using the stabilisation
  map, see e.g. \cite[Chapter 19.4]{may}.
  This means
  \[ H^k(X_{3,\infty}) = \set{ (a_i)\in \prod_{i=1}^\infty
    H^k(X_{3,i}) \sdel s^*(a_{i+1}) = a_i \;\forall i}. \]
  When considered in this way, the map $L_n^*$ is just the projection on
  the corresponding factor. But if we project to $H^k(X_{3,n})$ and
  get zero, we know that
  the only possible lift to $H^k(X_{3,n+1})$ is zero, since $s^*$ is
  injective. Continuing like this, we see that the only element of
  $H^k(X_{3,\infty})$ that can project to zero is the zero element, and
  hence $L_{n}^*$ is injective in this case.
  
  \subsubsection{$\lambda(\sigma) \geq 2$}
  
  If $\lambda(\sigma)=2$, we know $\sigma \in
  \set{(1\;2),(2\;3),(1\;2\;3)}$,
  $\sigma_{\emptyset} = \Id$ and that 
  \[ n \geq k + \lambda(\sigma) - 1 = k+1, \]
  giving 
  \[ n-1\geq k = k +\lambda(\Id)-1. \]
  Consider the following commutative diagram:
  \[ \xymatrix{ H^k(X_{3,\infty}(\sigma)) \ar[r]^{s^*}_{\cong}
    \ar[d]^{L_n^*} & 
    H^k(X_{3,\infty}) \ar[d]^{L_{n-1}^*} \\
    H^k(X_{3,n}(\sigma)) \ar[r]^{s^*} & H^k(X_{3,n-1}).
  } \]
  By the previous case, the map on the right is injective and so the
  composition $s^* \circ L_{n}^*$ is as well. Since the diagram
  commutes, this means that the left vertical map, $L_n^*$ must be
  injective, as desired.
  
  The last case has $\sigma_{\emptyset} = (2\;3)$ and proceeds exactly
  as the case $\lambda(\sigma) = 2$, using the injectivity that has
  already been proved for this permutation and that $\lambda((2\;3))$ is
  $\lambda(\sigma)-1$.
  
  \subsection{Surjectivity}
  
  We finish by proving that $L_n^*$ is surjective, so we assume $n \geq
  k+\lambda(\sigma)$. We have already
  worked with the stabilisation map $s^*$ and we now split into cases
  based on $\sigma$.
  
  \subsubsection{$\sigma = \Id$}
  
  We still know that the cohomology group of $X_{3,\infty}$ is the
  inverse limit of the
  cohomology groups of $X_{3,n}$ and that the map $L_n^*$ is the
  projection. But since we know all the maps 
  \[s^* : H^k(X_{3,j+1})\to H^k(X_{3,j}), \quad j \geq n, \]
  are surjective, we can hit anything in $H^k(X_{3,n})$ by choosing a
  lift in each $H^k(X_{3,j})$ for $j \geq n$ and use these to get an
  element of the inverse limit. Hence $L_n^*$ is surjective in this
  case.
  
  \subsubsection{$\lambda(\sigma) \geq 2$}
  
  For the other permutations, define groups $C_n(\sigma)$ as the
  cokernel of the map $L_n^*$, giving a short exact sequence:
  \[ \xymatrix{ 0 \ar[r] & H^k(X_{3,\infty}(\sigma))
    \ar[r]^{L_{n}^*} &
    H^k(X_{3,n}(\sigma)) \ar[r] &
    C_{n}(\sigma) \ar[r] &  0.} \]
  The maps
  \[s^* : H^k(X_{3,n}(\sigma)) \to H^k(X_{3,n-1}(\sigma_{\emptyset})) \]
  induces maps between these cokernels and gives a commuting diagram
  \[ \xymatrix{ 0 \ar[r] & H^k(X_{3,\infty}(\sigma))
    \ar[r]^{L_{n}^*}\ar[d]^-{s^*} &
    H^k(X_{3,n}(\sigma)) \ar[r]\ar[d]^-{s^*} &
    C_{n}(\sigma) \ar[r]\ar[d]^-{s^*} &  0 \\
    0 \ar[r] & H^k(X_{3,n}(\sigma_{\emptyset})) \ar[r]^{L_{n-1}^*} &
    H^k(X_{3,n-1}(\sigma_{\emptyset})) \ar[r] &
    C_{n-1}(\sigma_{\emptyset}) \ar[r] &  0. } \]
  If $\lambda(\sigma) = 2$, we know that $\sigma_{\emptyset} = \Id$, and
  if we combine this with the inequality
  \[ n \geq k+\lambda(\sigma) = k+2 \implies n-1 \geq
  k+\lambda(\Id), \]
  we conclude $C_{n-1}(\Id) = 0$ by the previous case. The
  maps not between cokernels are isomorphisms, since we have already
  proved
  injectivity and surjectivity of $s^*$, in the infinity case in
  Theorem \ref{thm:kollaps} and earlier in this proof in the case of $n
  < \infty$. Hence the map 
  \[ s^* : C_{n}(\sigma) \to C_{n-1}(\sigma_{\emptyset}) = 0 \]
  is an isomorphism by the 5-lemma, and so $L_n^*$ is surjective.
  
  In the final cases, $\lambda(\sigma) = 3$, we apply the above
  argument, using
  \[ \sigma_{\emptyset} = (2\; 3),\quad \lambda(\sigma_{\emptyset}) =
  1. \]
  Since
  \[ n \geq k+\lambda(\sigma) = k + 3 \implies n-1 \geq  k +
  \lambda((2\;3)), \]
  we conclude $C_{n-1}((2\;3))=0$ by the previous case, and the same
  arguments give us $C_{n}(\sigma) = 0$ and $L^*_n$ surjective.
  
  This shows that $I_k$ follows from $I_{k-1}$, and hence finishes the
  proof of the theorem.
\end{proof}

We will end this chapter with a quick application of this theorem to
$X_{3,3}$, since this is a group we have previously calculated.

\begin{example}
  Recall that
  \[ H^q(X_{3,3}) = \begin{cases}
    \Z & q = 0, \\
    \Z^{2} & q=1, \\
    \Z^{2} & q=2, \\
    \Z^{2} & q=3, \\
    \Z^{3} & q=4, \\
    \Z^{2} & q=5, \\
    \Z & q=6, \\
    0 & \text{otherwise}.
  \end{cases} \]
  If we consider the above theorem, we have $n = 3$ and
  $\lambda(\Id) = 1$, giving
  \[ 3 = n \geq k +\lambda(\sigma) = k + 1. \]
  So the theorem tells us the two groups $H^1(X_{3,3})$ and
  $H^2(X_{3,3})$ are isomorphic to the corresponding
  ones for
  $\Omega(\SU_3/\T^2)$.
  This fits with what is expected if we use the groups computed in
  \cite{grbic},
  \[ H_*(\Omega(\SU_3/\T^2)) \cong T(x_1,x_{2}) \otimes
  \Z[y_1,y_{2}]/I. \]
  In degree one we get the elements $x_1$ and $x_2$, and in degree two
  we get $y_1$ and $x_1 x_2$ as generators in the homology of the loop
  space, which
  correspond to the right dimensions of the cohomology groups by the
  universal coefficient theorem. Since the theorem also works for $n >
  3$, we conclude that for $q\in \set{1,2}$ and $n\geq 3$, we have
  \[ H^q(X_{3,n}) \cong \Z^2. \]
\end{example}

\begin{example}
  Another application of this theorem allows us to compute
  $H^k(X_{n,3})$ when $n \geq k+1$. We will do so inductively. We
  already know $H^1(X_{n,3}) \cong \Z^{n-1}$. If we want to compute
  $H^k(X_{n,3})$, we can use Lemma \ref{lem:transponering}, Lemma
  \ref{lem:reduktion}, the K\"unneth formula, and Theorem
  \ref{thm:kostab} to get the formula
  \begin{align*}
    \bigoplus_{r=0}^k H^r(X_{n,3}) \otimes H^{k-r}(\T^3) 
    &\cong H^k(Y_{n,3}) \\
    &\cong H^k(Y_{3,n}) \\
    &\cong \bigoplus_{r=0}^k H^r(X_{3,n}) \otimes H^{k-r}(\T^n) \\
    &\cong \bigoplus_{r=0}^k H^r(X_{3,\infty}) \otimes H^{k-r}(\T^n)
  \end{align*}
  By induction, all the groups are known except $H^k(X_{n,3})$, so
  this can be calculated by pure algebraic arguments. For example, if
  $n \geq 3$ we can compute $H^2(X_{n,3})$. The formula reduces to
  \[ H^2(X_{n,3}) \oplus \Z^{3n} \cong \Z^{2n+2+{n\choose 2}}. \]
  By manipulating this formula, we end up with
  \begin{align*}
    \rk_\Z H^2(X_{n,3}) &= 2 - n + {n\choose 2} \\
                        &= 1 - (n-1) + {n-1 \choose 1} + {n-1 \choose
                          2} \\
                        &= 1 + {n-1 \choose 2}.
  \end{align*}
  This fits with our expectation, since the relations on
  $H_2(\SUT{n})$
  tells us that it is generated by the elements $y_1$ and $x_p x_q$,
  $1\leq p<q\leq n-1$, so it has the same rank as $H^2(X_{n,3})$.
\end{example}

%%% Local Variables: 
%%% mode: latex
%%% TeX-master: "main"
%%% End: 
