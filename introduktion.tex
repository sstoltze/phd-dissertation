\chapter{Introduction}
\fixme{Referer til del a, sig at dele er taget derfra. Se
  http://phd.au.dk/fileadmin/grads.au.dk/ST/Rules\_and\_regulations/GSST\_Rules\_and\_Regulations.pdf,
  \$11.1}
\fixme{Sørg for at $T$ er $\mathrm{T}$ hele vejen. Samme for
  $\mathrm{B}$ og $\GL$ og $\SU$}
% En introduktion til afhandlingen, inklusive kapitel-oversigt og
% ideer bag. Her bør for eksempel indgå en forklaring af det der pjat
% med vektorbundter som Marcel har forsøgt at forklare før.
% Derudover nok også en oversigt over de vigtigste resultater der
% indgår.

%%%%% Previous
% \chapter{Introduction}

% % Noget om starten, hvorfor dette er interessant. Se bagside af blok
% % Noget om notation... eg \subset, D-0, hvad er D_p
% % Alting komplekst, så \GL_m giver mening etc. Måske også de
% % forskellige versioner af flag-mangfoldigheder.
% % Og noget med at T=\C^*....

% This document gives a description of a family of spaces $\X\subset
% \C^{mn}$, indexed by two natural numbers $m$ and $n$, and a quotient
% of these spaces, $Y_{m,n} = \X / T^n$. The spaces were
% originally defined as the zero-set of a section on a vector bundle of
% a Grassmann manifold, and were studied to understand the properties of
% families of Calabi-Yau manifolds. However, the spaces turned out to
% have other properties that made them interesting in their own
% right. The most important property is that they can be used to form a
% space that is homotopy equivalent to a space that appears in physics,
% namely the loop space $\Omega(\SUT{m})$. This hints at a way of
% extracting information about the loop space by working with the
% simpler spaces $Y_{m,n}$.

% The spaces have been studied by Marcel B\"okstedt, and some of his
% results are gathered in \cite{bokstedt}. In particular it includes the
% following theorem that covers the special case where $m=2$:
% \begin{theorem}
%   The homology of $Y_{2,n}$ is
%   \[ H_q(Y_{2,n}) =
%   \begin{cases}
%     \Z & 0 \leq q \leq n \\
%     0 & \text{otherwise}
%   \end{cases} \]
%   There is a stabilization map $s : Y_{2,n} \to Y_{2,n+1}$ that
%   induces an isomorphism
%   \[ s_* : H_q(Y_{2,n}) \to H_q(Y_{2,n+1}) \]
%   for $q \leq n$. There is also a map $f : Y_{2,n} \to
%   \Omega(\mathrm{SU_2}/T^1)$ that induces an isomorphism in homology in
%   degree $q\leq n$.
% \end{theorem}

% The aim of the following is to try and extend these results to the
% general case of $Y_{m,n}$, with $m\geq 2$.

% This progress report is divided into three main parts.
% The first chapter gives an introduction to the spaces we are
% interested in and lists some of the properties that will be used
% in the other chapters.
% The second chapter is mainly concerned with an explicit calculation of
% the cohomology of the spaces $X_{3,3}$ and $Y_{3,3}$. This is done by
% applying the spectral sequence coming from a filtration of $X_{3,3}$,
% computing the various relative cohomology groups needed, and
% finding the differentials of the sequence.
% The third and final chapter concerns the relation to the loop space
% described above. Here it is shown that by taking a limit of the spaces
% $Y_{m,n}$, we get a space homotopy equivalent to the loop space. The
% chapter concludes with a brief overview of plans for future studies.

%%% Local Variables: 
%%% mode: latex
%%% TeX-master: "main"
%%% End: 
