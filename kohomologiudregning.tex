
% Dette skal vist næsten bare kopieres direkte over. Skift X til Y og
% brug måske lidt mere tid på at sætte det op for kvotientrummene også
% så man kan bruge det direkte på dem i stedet for at skulle igennem
% de ukvotienterede.
%% Hvis det ikke er gjort i tidligere kapitler, så lav udregningen af
%% H^1 generelt her et sted.


%%%%% Previous
\chapter{Cohomology calculations for $Y_{3,3}$}
\label{chap:udregninger}

In this section we will calculate the cohomology of the space
$Y_{3,3}$, which is the set:
\[ Y_{3,3} = \set{
  \begin{pmatrix}
    a & b & g \\
    c & d & h \\
    e & f & i
  \end{pmatrix} \in \C^9 \sdel 
  \begin{matrix}
    a\neq 0,\, ad-bc \neq 0, \\
    adi+beh+cfg-deg-bci-afh \neq 0, \\
    di-fh \neq 0,\, i \neq 0
  \end{matrix} } \]

The way we will calculate this is by using the spectral sequence for
a filtration of $Y_{3,3}$. Some of the calculations have
been done in Sage and the relevant source code can be found in
Appendix \ref{appendix:sage}.

\section{A filtration on $\Y$}
\label{sec:filtration}

Before we start the calculation, we need some general setup. To work
with the space $Y_{m,n}$ and compute the cohomology, it will
be beneficial to build it up from simpler subspaces. This is done by
considering a filtration.

For $0 \leq p \leq m-1$, let $F_p$ be the subset
\[ F_p=\set{ (a_1,\dots,a_n) \in Y_{m,n} \mid a_n \text{ has at most } p
  \text{ zeroes}.} \]
$F_p$ is open in $\X$ since we stay inside the subspace when we
vary the coordinates of a point $A\in F_p$, as long as we do not
introduce new zeroes in the last column.
Together, the subspaces form an increasing sequence:
\[ F_0 \subset F_1 \subset \dots \subset F_{m-1} = Y_{m,n} \]

The general method of computing the homology and cohomology of
$Y_{m,n}$ is by using the spectral sequence of this
filtration, as described in \cite{hatcherss} and
\cite{mccleary}. Here we will focus on
cohomology, but similar arguments could of course be applied to
homology.

The spectral sequence has $E_1$-page given by
\[ E_1^{p,q} = H^{p+q}(F_p,F_{p-1}) \]
with differentials given by the boundary map in the long exact
sequence of the triple $(F_{p+1},F_p,F_{p-1})$:
\[ d_1 : H^{p+q}(F_p,F_{p-1}) = E_1^{p,q} \to E_1^{p+1,q} =
H^{p+q+1}(F_{p+1},F_p) \]

To use the spectral sequence, we need to compute the cohomology of the
pair $(F_p,F_{p-1})$. To calculate this we will use excision, so we
would like to find an open subset of $F_p$ that contains the
complement of $F_{p-1}$. For any choice of exactly $p$ 
different indices $\set{v_1,\dots,v_p}\subset \set{1,\dots,m-1}$,
define $Y_{\set{v_1,\dots,v_p}}$ to be the points $(a_1,\dots,a_n)\in
F_p$ which has $(a_n)_{v_i} = 0$ for all $i$. Since $\X$ is open, we
can get an open set in $F_p$ by inserting a small disc on these
zero-entries. More precisely, the set we get is
\[ Y_{\set{v_1,\dots,v_p}} \times D_p \cong \set{ A\in F_p \sdel 
  \begin{matrix} \displaystyle 
    \size{((a_n)_{v_1},\dots,(a_n)_{v_p})} <
    \min_{i\not\in \set{v_1,\dots,v_p}} \size{(a_n)_i}, \\
    \left(a_1,\dots,a_{n-1},a_n-t\sum_{i} (a_n)_{v_i}e_{v_i}\right)\in
    F_p \forall t \in [0,1]
  \end{matrix} } \]
This is an open set in $F_p$, with $Y_{\emptyset} = F_0$, and it
contains all points $(a_1,\dots,a_n)\in F_p$ where the entries of
$a_n$ indicated by $\set{v_1,\dots,v_p}$ are zero. For a different
choice of indices, the condition that the radius of the disc is
smaller than the norm of the non-zero entries in $a_n$ ensures
disjointness, so the intersection is empty:
\[ \left(Y_{\set{v_1,\dots,v_p}}\times D_p\right) \cap \left(
  Y_{\set{u_1,\dots,u_p}}\times D_p\right) = \emptyset \quad \text{if }
\set{v_1,\dots,v_p} \neq \set{u_1,\dots,u_p} \]

The intersection of $\X\times D_p$ and $F_{p-1}$ corresponds to
removing 0 from the disc, since one of the entries inserted on the
last column has to be non-zero.
\[ \left(Y_{\set{v_1,\dots,v_p}}\times D_p \right) \cap F_{p-1} =
Y_{\set{v_1,\dots,v_p}} \times \left(D_p-0\right) \]
This means that the disjoint union of all of these form an open set in
$F_p$ with
\[ F_p = F_{p-1}\cup \left(\coprod_{\set{v_1,\dots,v_p}}
  Y_{\set{v_1,\dots,v_p}}\times D_p\right) \]
and by excision, the inclusion of pairs
\[ \xymatrix{ \displaystyle \coprod_{\set{v_1,\dots,v_p}}
  Y_{\set{v_1,\dots,v_p}}\times (D_p,D_p-0) \ar@{^(->}[r] &
  (F_p,F_{p-1}) } \]
induces an isomorphism of cohomology groups,
\begin{align*}
  E_1^{p,q} &= H^{p+q}(F_p,F_{p-1}) \\
            &\cong H^{p+q}\left(\coprod_{\set{v_1,\dots,v_p}}
              Y_{\set{v_1,\dots,v_p}}\times (D_p,D_p-0)\right) \\   
            &\cong \bigoplus_{\set{v_1,\dots,v_p}}
              H^{p+q}(Y_{\set{v_1,\dots,v_p}}\times (D_p,D_p-0)) \\   
            &\cong \bigoplus_{\set{v_1,\dots,v_p}}
              H^{q-p}(Y_{\set{v_1,\dots,v_p}})\otimes H^{2p}(D_p,D_p-0)
\end{align*}
The last isomorphism is given by the K\"unneth formula, and the fact
that the pair $(D_p,D_p-0)$ only has non-zero cohomology in degree
$2p$:
\[ H^q(D_p,D_p-0) \cong \widetilde H^{q-1}(D_p-0) = \widetilde
H^{q-1}(S^{2p-1}) =
\begin{cases}
  \Z & q = 2p \\
  0 & \text{otherwise}
\end{cases} \]
In particular, it follows that
\[ E_1^{p,q} = 0 \]
for $q < p$.

\section{The first cohomology of $X_{m,n}$}
\label{sec:firstcohom}

We would like to use the spectral sequence to compute the first
cohomology group of $X_{m,n}$ for any $m$ and $n$. As discussed at the
end of Section \ref{sec:rum}, we will do this by computing
$H^1(Y_{m,n})$ and
then using Lemma \ref{lem:reduktion} to calculate the cohomology of
$X_{m,n}$ via factoring out the extra free group from the torus
in the isomorphism
\[ H^1(Y_{m,n}) \cong H^1(X_{m,n}) \oplus H^1(\T^n). \]

From the theory
underlying the spectral sequence, see e.g. \cite[Chapter
1]{hatcherss}, we get that there is a subgroup
\[ 0 \subset H \subset H^1(Y_{m,n}) \]
and isomorphisms
\begin{align*}
  E_\infty^{0,1} &\cong H^1(Y_{m,n})/H, \\
  E_\infty^{1,0} &\cong H.
\end{align*}
So to calculate $H^1(Y_{m,n})$, we have to calculate the two groups
$E_\infty^{1,0}$ and $E_\infty^{0,1}$ and see what groups fit
into the short exact sequence
\[ 0 \to E^{1,0}_\infty \to H^1(Y_{m,n}) \to E^{0,1}_\infty \to 0. \]

Drawing the relevant part of the first page of the spectral sequence,
we are considering the diagram illustrated in Figure
\ref{fig:foerste}.
\begin{figure}[ht]
  \[ \xy
  (5,4); (5,-13); **\dir{-}; (5,4); *\dir{>};
  (5,-13); (73,-13); **\dir{-}; *\dir{<};
  (0,-15)*{q}; (5,-17)*{p};
  (1,-17); (4,-14); **\dir{-};
  (0,0)\xymatrix@R=1em@C=1.7em{
    1 & H^1(F_0) \ar[r]& H^2(F_1,F_0) \ar[r] & H^3(F_2,F_1) \\
    0 & H^0(F_0) \ar[r] & H^1(F_1,F_0) \ar[r] & H^2(F_2,F_1) \\
    & 0 & 1 & 2
  } \endxy \]
  \caption{The relevant part of $E_1$ for computing $H^1(Y_{m,n})$.}
  \label{fig:foerste}
\end{figure}
But by writing out our specific case, we can see that the the groups
$H^1(F_1,F_0), H^2(F_2,F_1)$ and $H^3(F_2,F_1)$ are all zero, since $q
< p$, so the diagram simplifies to Figure \ref{fig:anden}.
\begin{figure}[ht]
  \[ \xy
  (5,4); (5,-13); **\dir{-}; (5,4); *\dir{>};
  (5,-13); (75,-13); **\dir{-}; *\dir{<};
  (0,-15)*{q}; (5,-17)*{p};
  (1,-17); (4,-14); **\dir{-};
  (0,0)\xymatrix@R=1em@C=1.7em{
    1 & H^1(Y_{m,n-1}\times \T^m) \ar[r]& H^2(F_1,F_0) & 0 \\
    0 & H^0(Y_{m,n-1}\times \T^{m}) & 0 & 0 \\
    & 0 & 1 & 2
  } \endxy \]
  \caption{The simplified $E_1$.}
  \label{fig:anden}
\end{figure}
From the figure, we can see that the group $E_\infty^{1,0}$
must be zero, so the exact sequence we are considering gives an
isomorphism
\[ \xymatrix{0 \ar[r] & H^1(Y_{m,n}) \ar[r]^>>>>>{\cong} & E^{0,1}_\infty
  \ar[r] & 0}. \]

By doing explicit calculations for small $m$ and $n$, as in Example
\ref{ex:n=1}, \ref{ex:22}, and \ref{ex:32}, we are led to
the following theorem.

\begin{theorem}
  \label{thm:forste}
  The first cohomology group of $Y_{m,n}$ is isomorphic to $\Z^{m+n-1}$.
\end{theorem}
\begin{proof}
  We will proceed by induction on $n$. By looking at Example
  \ref{ex:n=1}, we see that
  \[ H^1(Y_{m,1}) \cong H^1(\T^{m}) \cong \Z^m = \Z^{m+1-1} \]
  as desired.

  Now assume the theorem is correct for 
  \[ H^1(Y_{m,n-1}) \cong \Z^{m+n-2}. \]
  By excision, we know that the group $H^2(F_1,F_0)$ is given by
  \[ H^2(F_1,F_0) \cong \bigoplus_{i=1}^{m-1} H^0(Y_{m,n-1}\big((i\;
  i+1)\big) \otimes H^2(D,D-0) \cong \Z^{m-1}. \]
  \begin{figure}[ht]
    \[ \xy
    (5,4); (5,-13); **\dir{-}; (5,4); *\dir{>};
    (5,-13); (61,-13); **\dir{-}; *\dir{<};
    (0,-15)*{q}; (5,-17)*{p};
    (1,-17); (4,-14); **\dir{-};
    (0,0)\xymatrix@R=1em@C=1.7em{
      1 & \Z^{m+n-2}\oplus\Z^m \ar[r]& \Z^{m-1} & 0 \\
      0 & \Z & 0 & 0 \\
      & 0 & 1 & 2
    } \endxy \]
    \caption{$E_1$, assuming the induction hypothesis.}
    \label{fig:tredje}
  \end{figure}
  This simplifies Figure \ref{fig:anden} to Figure
  \ref{fig:tredje}. Since the groups involved will never change after
  turning the page to $E_2$, we see that we only have to compute the
  group
  \[ E_\infty^{0,1} = E_2^{0,1} = \ker \left( d_1 : E_1^{0,1} \to
    E_1^{1,1} \right). \]
  The differential $d_1$ is given by the differential in the long
  exact sequence for the pair $(F_1,F_0)$, which is the composition
  \[ d_1 : \xymatrix{ H^1\left(Y_{m,n-1}\times \T^m\right)
    \ar[r]^<<<<<<<<<{\oplus\imath^*}  & \bigoplus_{i=1}^{m-1}
    H^1\left(Y_{m,n-1}\big((i\; i+1)\big)\times \T^{m-1} \times
      (D-0)\right)  \\
    \hspace{5em}\ar[r]^<<<<<{\oplus\delta} & \bigoplus_{i=1}^{m-1}
    H^2\left(Y_{m,n}\big((i\; i+1)\big)\times \T^{m-1} \times
      (D,D-0)\right).}\] \fixme{Få pilen til at sidde ordentligt}
  The first map is induced by the inclusions
  \[ \imath : Y_{m,n-1}\big((i\; i+1)\big) \times \T^{m-1} \times (D-0)
  \to Y^{m,n-1}\times \T^m, \] 
  given by first using the inverse of the homeomorphism from Lemma
  \ref{lem:permutation}, given by the indices $I = (1 < 2 < \dots <
  i-1 < i+1 < \dots m)$,
  \[ Y_{m,n}^I \to Y_{m,n+1}\big((i \; i+1)\big) \times \T^{m-1}, \]
  and then inserting the coordinate from $(D-0)$ on the $i$'th
  coordinate of the last column.
  The second map consists of the differentials from the long exact
  sequences of the pairs
  \[ Y_{m,n-1}\big((i\; i+1)\big) \times \T^{m-1}\times (D,D-0). \]
  We want to see that $d_1$ is surjective, so we will compute the
  image of the element 
  \[ \alpha_j = 1\otimes\dots\otimes 1\otimes [T] \otimes 1 \otimes
  \dots
  \otimes 1 \in H^1(Y_{m,n-1}\times \T^{m}),\] 
  where the class $[T]\in H^1(\T)$ is in the $j$'th tensor coordinate.
  We will consider each inclusion separately. If $i \neq j$, the
  inclusion will not touch the $j$'th coordinate of $\T^m$, so on
  cohomology classes it will map $\alpha_j$ to an element that is of
  the form
  \[ \imath^*(\alpha_j) = \widetilde{\alpha_j}\otimes 1 \in
  H^*(Y_{m,n-1}\big((i\; i+1) \big) \times \T^{m-1}) \otimes
  H^*(D-0) \]
  And since $\delta$ is given by \fixme{Bør nok også argumenteres for
    et sted}
  \[ \delta(\beta) =
  \begin{cases}
    \widetilde{\beta}\otimes [(D,D-0)] & \text{if } \beta =
    \widetilde{\beta}\otimes [D-0] \\
    0 & \text{otherwise}
  \end{cases}, \]
  we see that $\delta(\imath^*(\alpha_j)) = 0$ in this case. If $i =
  j$, then the same considerations give us that
  \[ \delta\imath^*(\alpha_j) = 1 \otimes [(D,D-0)] \in
  H^2(Y_{m,n-1}\big((i\; i+1) \big) \times \T^{m-1}\times (D,D-0)). \]
  The group $H^2(F_1,F_0)$ is generated by exactly these elements, so
  the map $d_1$ is surjective. Since the groups involved are all free,
  we know the kernel is also free and we can calculate the rank as the
  rank of the domain minus the rank of the codomain, giving us
  \[ H^1(Y_{m,n}) \cong E_{\infty}^{0,1} = \ker d_1 \cong
  \Z^{m+n-2+m-(m-1)} = \Z^{m+n-1}. \]
\end{proof}

An immediate consequence is the following result.

\begin{corollary}
  The first cohomology group of $X_{m,n}$ is $\Z^{m-1}$.
\end{corollary}
\begin{proof}
  By using the K\"unneth formula and the previous theorem, we get an
  isomorphism
  \[ H^1(X_{m,n}) \oplus H^1(\T^n) \cong  H^1(Y_{m,n}) \cong
  \Z^{m+n-1}. \]
  Since any torsion in $H^1(X_{m,n})$ would show up in $H^1(Y_{m,n})$,
  it must be free. We can calculate the rank from the above and get
  \[ H^1(X_{m,n}) \cong \Z^{m-1}. \]
\end{proof}

To make this abstract isomorphism slightly more concrete, we will
realise it as a map between spaces. Consider the image of an element
$v \in Y_{m,1}$ in $Y_{m,n}$ under the stabilisation map. This has the
form 
\[ s^{n-1}(v) = s^{n-1}
\begin{pmatrix}
  v_1 \\
  \vdots \\
  v_m
\end{pmatrix} =
\begin{pmatrix}
  \vrule & \vrule & & \vrule \\
  L^{n} v & L^{n-1}e_1 & \dots & Le_1 \\
  \vrule & \vrule & & \vrule
\end{pmatrix},
\]
where $L$ is the lower triangular matrix that defines the
stabilisation map, see Definition \ref{def:stabilisering}. Computing the
determinants gives
\[ \Det(s^{n-1}(v)) = (\pm v_1,\dots,\pm v_m, \pm 1, \dots,\pm 1), \]
with the signs dependent on $n$ and $m$. Using the action of
$\T^{n}$ to change the last $n-1$ determinants as in
\fixme{find ud af hvad der skal refereres til
  her}, we can get an element of $Y_{m,n}$ with any prescribed
determinant. Altogether this defines a lift, $f$, of the
determinant map,
\[ \xymatrix{ \T^{m+n-1} \ar[r]^f \ar@/_1.5pc/[rr]_\Id & Y_{m,n}
  \ar[r]^<<<<<\Det & \T^{m+n-1}}. \]
By taking first cohomology, we get maps
\[ \xymatrix{ H^1(Y_{m,n}) \ar@/^/[r]^{f^*} & H^1(\T^{m+n-1})
  \ar@/^/[l]^{\Det^*} }, \]
with 
\[ f^* \circ \Det^* = \Id. \]
This shows that $\Det^*$ is injective and $f^*$ is surjective. But by
the previous theorem, $f^*$ is a map between free abelian groups of
the same rank, hence it must also be injective and so an
isomorphism with inverse $\Det^*$. This shows that the determinant map
is an isomorphism on the first cohomology groups.
This result immediately extends to the quotiented spaces, with
\fixme{Sørg for at $\TT{m+n-1}{n}$ er introduceret før. Det er ikke
  den oplagte virkning...}
\[ \Det^* : H^1(\TT{m+n-1}{n}) \to H^1(X_{m,n}) \]
an isomorphism. Furthermore it shows that the stabilisation maps on
degree one cohomology are surjective, since the stabilisation maps
\[ s: Y_{m,n} \to Y_{m,n+1} \]
change the signs of some determinants and add an extra determinant
that is always plus or minus one. So if the class $d_i \in
H^1(Y_{m,n+1})$ represents the image of moving once around zero in the
$i$'th coordinate of $\T^{n+1}$, then
\[ s^*(d_i) =
\begin{cases}
  \pm d_i & 1 \leq i \leq n \\
  0 & \text{otherwise}
\end{cases} \] \fixme{Muligvis skift så vi bruger $x_i$ eller $t_i$ eller
  whatever der nu giver mening når jeg har skrevet kapitlet med
  udregninger}



\section{The spectral sequence of $Y_{3,3}$}
\label{sec:ss33}

We can apply the above filtration to get a spectral sequence of
$X=Y_{3,3}$. The groups we need to compute for the first page are:
\[ E_1^{p,q} \cong
\begin{cases}
  H^{q}(Y_\emptyset) & p=0 \\
  H^{q+1}(Y_{\set{1}} \times (D,D-0)) \oplus H^{q+1}(Y_{\set{2}} \times
  (D,D-0)) & p=1 \\
  H^{q+2}(Y_{\set{1,2}}\times (D_2,D_2-0)) & p = 2 \\
  0 & \text{otherwise}
\end{cases} \]
The four spaces that appear will be handled separately.

\subsection{The cohomology of $Y_{\set{1,2}}$}

Consider the space $Y_{\set{1,2}}$. This consists of the elements of
$Y_{3,3}$ that have two zeroes in the last column:
\[ Y_{\set{1,2}} = \set{
  \begin{pmatrix}
    a & b & 0 \\
    c & d & 0 \\
    e & f & i
  \end{pmatrix} \sdel
  \begin{matrix}
    a \neq 0, ad-bc \neq 0 \\
    adi-bci \neq 0 \\
    di \neq 0, i \neq 0
  \end{matrix}
} \]

The space is homeomorphic to $(\C^*)^3\times \C^2
\times \set{(b,c)\in\C^2 \delim \vert \delim bc\neq 1}$, with the
homeomorphism given by
\begin{align*}
  Y_{\set{1,2}} &\to (\C^*)^3\times \C^2
  \times \set{(b,c)\in\C^2 \delim \vert \delim bc\neq 1} \\
  \begin{pmatrix}
    a & b & 0 \\
    c & d & 0 \\
    e & f & i
  \end{pmatrix} &\mapsto \left(a,d,i,\frac{e}{i},\frac{f}{i},
    \frac{b}{a}, \frac{c}{d}\right)
\end{align*}

We already computed the cohomology of $Y =\set{(b,c)\delim \vert\delim
  bc\neq 1}$ in Example \ref{ex:22}, so we can calculate the
cohomology of $Y_{\set{1,2}}$:
\[ H^q(Y_{\set{1,2}}) =
\begin{cases}
  \Z & q = 0 \\
  \Z^4 & q = 1 \\
  \Z^7 & q = 2 \\
  \Z^7 & q = 3 \\
  \Z^4 & q = 4 \\
  \Z & q = 5 \\
  0 & \text{otherwise}
\end{cases} \]

Since we know the product structure from the K\"unneth formula, we can
name the generators and get the ring structure,
\[ H^*(Y_{\set{1,2}}) = \bigwedge\nolimits^*[a_{00},d_{00},i_{00}]\otimes
H^*(Y) \]
The two generators of $Y$ will be denoted $y_{00,1}$ for the degree
one generator and $y_{00,2}$ for the degree two generator. All other
generators have degree one. The ring
structure is the usual one on a tensor product, with the added
relation $y_{00,1}\cup y_{00,2} = 0$. The element $y_{00,1}\in H^1(Y)$
is given by the map
\[ Y\ni (b,c) \mapsto 1-bc\in \C^* \]
in cohomology, since there is another map
\[ \C^* \ni \l \mapsto (1,1-\l) \in Y \]
and the composition is the identity on $\C^*$. Applying $H^1$ shows
that the two maps give isomorphisms in degree 1.


\subsection{The cohomology of $Y_{\set{1}}$}

In $Y_{\set{1}}$ we have the first entry of the last column equal to
zero, which is the space
\[ Y_{\set{1}} = \set{
  \begin{pmatrix}
    a & b & 0 \\
    c & d & h \\
    e & f & i
  \end{pmatrix} \sdel
  \begin{matrix}
    a \neq 0, ad-bc \neq 0 \\
    adi + beh - bci - afh \neq 0 \\
    di -fh \neq 0, h \neq 0, i \neq 0 
  \end{matrix}
} \]

This space is homeomorphic to $(\C^*)^5 \times \C\times Y$:
\[ \begin{pmatrix}
  a & b & 0 \\
  c & d & h \\
  e & f & i
\end{pmatrix} \mapsto \left(
a, i, h, \frac{d}{h}-\frac{f}{i}, \frac{d}{h}-\frac{bc}{ah},
\frac{e}{i}, \frac{bhi}{adi-afh},\frac{c}{h}-\frac{e}{i}
\right) \] 
Since the cohomology of $Y$ is already known, applying the K\"unneth
formula gives
\[ H^q(Y_{\set{1}}) =
\begin{cases}
  \Z & q = 0 \\
  \Z^6 & q = 1 \\
  \Z^{16} & q = 2 \\
  \Z^{25} & q = 3 \\
  \Z^{25} & q = 4 \\
  \Z^{16} & q = 5 \\
  \Z^6 & q = 6 \\
  \Z & q = 7 \\
  0 & \text{otherwise}
\end{cases} \] 
As before, the ring structure is given by
\[ H^*(Y_{\set{1}}) =
\bigwedge\nolimits^*[a_{01},d_{01},f_{01},h_{01},i_{01}] \otimes H^*(Y)\]

\subsection{The cohomology of $Y_{\set{2}}$}

The space is
\[ Y_{\set{2}} = \set{
  \begin{pmatrix}
    a & b & g \\
    c & d & 0 \\
    e & f & i
  \end{pmatrix} \sdel
  \begin{matrix}
    a \neq 0, ad-bc \neq 0 \\
    adi +cfg -bci - deg \neq 0 \\
    di \neq 0, g\neq 0, i \neq 0 \end{matrix}
} \]
which can also be identified with $(\C^*)^5\times \C\times Y$ using
the homeomorphism
\[ \begin{pmatrix}
  a & b & g \\
  c & d & 0 \\
  e & f & i
\end{pmatrix} \mapsto \left(\frac{a}{g},d,i,g, 1+\frac{cfg}{adi}-
  \frac{eg}{ai}-\frac{bc}{ad}, \frac{f}{i},\frac{b}{a}, \frac{c}{d}
\right)\]
Since we already know the cohomology of this space, we know
\begin{align*}
  H^q(Y_{\set{2}}) = 
  \left(\bigwedge\nolimits^*[a_{10},d_{10},f_{10},g_{10},i_{10}]\otimes
  H^*(Y)\right)^q &=
  \begin{cases}
    \Z & q = 0 \\
    \Z^6 & q = 1 \\
    \Z^{16} & q = 2 \\
    \Z^{25} & q = 3 \\
    \Z^{25} & q = 4 \\
    \Z^{16} & q = 5 \\
    \Z^6 & q = 6 \\
    \Z & q = 7 \\
    0 & \text{otherwise}
  \end{cases}
\end{align*}

\subsection{The cohomology of $Y_{\emptyset}$}

This space is slightly more complicated than the others. The space
consists of the elements in $Y_{3,3}$ where all entries of the last
vector are non-zero.
\[ Y_{\emptyset} = \set{
  \begin{pmatrix}
    a & b & g \\
    c & d & h \\
    e & f & i
  \end{pmatrix} \sdel
  \begin{matrix}
    a \neq 0, ad-bc \neq 0, di-fh \neq 0, \\
    adi+beh+cfg-deg-bci-afh \neq 0, \\
    g\neq 0, h\neq 0, i\neq 0
  \end{matrix} }\]
The space is homeomorphic to $(\C^*)^7\times \set{ (x,y,z)\in \C^3
  \mid y-z-xyz = 1}$, with the homeomorphism given by
\begin{align*}
  \begin{pmatrix}
    a & b & g \\
    c & d & h \\
    e & f & i
  \end{pmatrix} \mapsto \bigg(&g, h, i, \frac{a}{g},
    \frac{f}{i}-\frac{d}{h},\\
  &D=\frac{ad-bc}{ah}, \\
  &C=\frac{c}{h}-\frac{a}{g}+\frac{cdi-deh}{fh^2-dhi} +
  \frac{beh-bci}{fgh-dgi},\\
  &\left.\frac{bC}{aD},\frac{c}{hC}-\frac{a}{gC},
    \frac{D}{C}\frac{eh-ci}{fh-di}\right)
\end{align*}

To get the cohomology, we refer back to Example \ref{ex:32}, where we
computed the cohomology of
\[ Z=\set{(x,y,z)\in\C^3 \mid y-z-xyz = 1}, \]
and found it to be the same as the cohomology of $S^2$. The generator of
$H^2(Z)$ will be called $z_{11}$ with $z_{11}^2 = 0$. If we apply
this, we get
\[ H^q(Y_{\emptyset}) =
\begin{cases}
  \Z & q = 0, \\
  \Z^7 & q = 1, \\
    \Z^{22} & q = 2, \\
    \Z^{42} & q = 3, \\
    \Z^{56} & q = 4, \\
    \Z^{56} & q = 5, \\
    \Z^{42} & q = 6, \\
    \Z^{22} & q = 7, \\
    \Z^{7} & q = 8, \\
    \Z & q = 9, \\
    0 & \text{otherwise}.
  \end{cases} \] 
and the generators are
\[ H^*(Y_{\emptyset}) =
\bigwedge\nolimits^*[a_{11},d_{11},g_{11},h_{11},i_{11},C_{11},D_{11}]
\otimes H^*(Z) \]

\subsubsection{A note on the cohomology of $Z$}

The space $Z$ defined above is a manifold, which we will prove by
applying the implicit function theorem. Consider the formula defining
$Z$,
\[ (x,y,z) \mapsto y-z-xyz-1. \]
Taking partial derivatives gives the three functions
\begin{align*}
  (x,y,z) &\mapsto yz, \\
  (x,y,z) &\mapsto 1-xz, \\
  (x,y,z) &\mapsto -1-xy.
\end{align*}
If the first function is zero, then at least one of the two others
must be non-zero. By the implicit function theorem, the coordinate
that has non-zero derivative can be written as a smooth function of
the other coordinates. This shows that $Z$ is locally the graph of a
smooth function, so it is a smooth manifold.

Later on it will be necessary to have a description of $H^2(Z)$. This
is done by considering the subspace
\[ R = \set{(x,y,z) \in \R^3 \sdel x > 0, y < 0, xy < -1, z =
  \frac{y-1}{1+xy}>0 } \subset Z. \]
This space is closed in $Z$ since any sequence in $R$ that converges
in $Z$ must also converge in $R$.
Consider a subbundle $NR \cong R\times D$ of the normal bundle of $R$
in the tangent bundle $TZ$, such that the exponential map is a
diffeomorphism on $NR$. This gives us a diagram
\[ \xymatrix{ H^2(NR,NR_0) & H^2(Z,Z-R) \ar[l]_\cong \ar[r]^>>>>>{j^*} &
  H^2(Z), } \]
where $NR_0$ is $NR$ without the zero section, $NR_0 \cong
R\times(D-0)$. The isomorphism is given by excision, since $Z = NR
\cup (Z-R)$. The space $R$ is homeomorphic to $\R^2$, so we have 
\[ H^2(Z,Z-R) \cong H^2(NR,NR_0) \cong H^2(R\times (D,D-0)) \cong
\Z. \] 
From the diagram, $j^*(1)$ gives us an element of $H^2(Z)$. To see
that this is a generator, consider the torus 
\[ \widehat \T = \set{(x,y,z)\in Z \mid \size{x} = \size{y} = 2 }
\subset Z. \]
This intersects $R$ in exactly one point, namely $p=(2,-2,1)$. By
taking the intersection with $\widehat T$ we can extend the previous
diagram,
\[ \xymatrix{ H^2(NR,NR_0) \ar[d]^{i^*} & H^2(Z,Z-R) \ar[d]^{i^*}
  \ar[l]_<<<<<\cong \ar[r]^>>>>>{j^*} & H^2(Z) \ar[d]^{i^*} \\
  H^2(p\times (D,D-0)) & H^2(\widehat \T,\widehat \T-p)
  \ar[r]^>>>>>{\cong} \ar[l]_>>>>>\cong & H^2(\widehat \T).
} \]
The left $i^*$ is an isomorphism since the cohomology in degree two is
given by the pair $(D,D-0)$, which does not change under the
inclusion. This shows that the inclusion $\widehat \T \to Z$ is an
isomorphism in cohomology, so $j^*(1)$ is a generator of
$H^2(Z)$. Taking the same diagram with cohomology replaced with
homology reverses all arrows but is otherwise exactly the same. This
shows that to calculate the homology class of a cycle $\sigma$ in
$H_2(Z)$, we can count the number of points of intersection between
$\sigma$ and $R$ with orientation, since each of these will contribute
either $1$ or $-1$, depending on orientation, to the homology class in
$(Z,Z-R)$.

There is a smiliar argument with the space $Y$ from before and the
subspace $R_Y = \set{(b,c) \in \R^2 \mid b > 0, c > 0, bc > 1}$. In
particular, the generator of $H_2(Y)$ can be taken as
$i_*\left([\widehat T]\right)$, where $[\widehat
  T]$ is a generator of $H_2(\widehat \T)$ for the torus $\widehat \T
  = \set{(b,c)\mid \size{b}=\size{c} = 2}$.

\subsection{The first page}

With all of the above, we can draw the $E_1$-page of the spectral
sequence. Remember that this is
\[ E_1^{p,q} \cong
\begin{cases}
  H^{q}(Y_\emptyset) & p=0 \\
  H^{q+1}(Y_{\set{1}} \times (D,D-0)) \oplus H^{q+1}(Y_{\set{2}} \times
  (D,D-0)) & p=1 \\
  H^{q+2}(Y_{\set{1,2}}\times (D_2,D_2-0)) & p = 2 \\
  0 & \text{otherwise}
\end{cases} \]

By inserting the groups that were calculated above into this, we get
Figure \ref{fig:e1} with the non-zero differentials moving directly to
the right. To get the $E_2$-page of the spectral sequence, we need to
compute the homology of the differentials.

\begin{figure}[ht]
  \[ \xy
  (5,4); (5,-80); **\dir{-}; (5,4); *\dir{>};
  (5,-80); (42,-80); **\dir{-}; *\dir{<};
  (0,-82)*{q}; (5,-84)*{p};
  (1,-84); (4,-81); **\dir{-};
  (0,0)\xymatrix@R=1em@C=1.7em{
    9 & \Z & 0 & 0 \\
    8 & \Z^7 \ar[r]& \Z^2 & 0 \\
    7 & \Z^{22} \ar[r]& \Z^{12} \ar[r]& \Z \\
    6 & \Z^{42} \ar[r]& \Z^{32} \ar[r]& \Z^{4} \\
    5 & \Z^{56} \ar[r]& \Z^{50} \ar[r]& \Z^{7} \\
    4 & \Z^{56} \ar[r]& \Z^{50} \ar[r]& \Z^{7} \\
    3 & \Z^{42} \ar[r]& \Z^{32} \ar[r]& \Z^{4} \\
    2 & \Z^{22} \ar[r]& \Z^{12} \ar[r]& \Z \\
    1 & \Z^{7} \ar[r]& \Z^{2} & 0 \\
    0 & \Z & 0 & 0 \\
    & 0 & 1 & 2
  } \endxy \]
  \caption{The $E_1$-page with differentials.}
  \label{fig:e1}
\end{figure}

\section{Differentials on $E_1$}
\label{sec:orientation}

By definition, the differentials are the boundary maps of the long
exact sequence of the triple $(F_{p+1},F_p,F_{p-1})$,
\[ d_1 : E_1^{p,q} = H^{p+q}(F_p,F_{p-1}) \to H^{p+q+1}(F_{p+1},F_p) =
E_1^{p+1,q} \]
We will consider the two cases $p=0$ and $p=1$ separately.

\subsection{First column}

For $p = 0$, we are looking at the boundary map of the pair
$(F_1,F_0)$. Consider the inclusion of pairs,
\[ i : Y_{\set{1}}\times (D,D-0) \sqcup Y_{\set{2}}\times (D,D-0) \to
(F_1,F_0) \]
This is an isomorphism in cohomology and gives a map between the long
exact sequences that makes the following diagram
commute:
\[ \xymatrix{
H^{p+q}(F_0) \ar[r]^>>>>>>>>>>>>>>>>>>>>{d_1} \ar[d]^{i^*} &
H^{p+q+1}(F_1,F_0) \ar[d]^{i^*}_{\cong} \\
H^{p+q}((Y_{\set{1}}\sqcup Y_{\set{2}})\times (D-0))
\ar[r]^>>>>>{\delta} & H^{p+q+1}((Y_{\set{1}}\sqcup Y_{\set{2}})\times
(D,D-0)) } \]
The bottom map is the boundary map from the long exact sequence of the
pair $(D,D-0)$. This map is well-known and will be treated at the end
of this section, so the only missing part is to calculate the
inclusion $i^*$. Since we are working with a disjoint union, we only
need to calculate the inclusion on each factor.

\subsubsection{The inclusion $Y_{\set{1}}\times (D-0) 
  \hookrightarrow Y_\emptyset$}

We can write the inclusion out in coordinates as
\begin{align*}
  (\C^*)^5\times Y \times (D-0) \longrightarrow
  (&\C^*)^7\times Z \\
  (a,d,f,h,i,b,c,g) \mapsto \bigg(&g,h,i,\frac{a}{g},-f,d,\\
  & C =
  c-\frac{a}{g}-\frac{cd}{f}-bc^2+\frac{abc}{g}, \\
  & \frac{bfC}{d},\frac{c}{C}-\frac{a}{g C},\frac{cd}{fC}\bigg) 
\end{align*}

On the generators, this is given by
\begin{align*}
  i^* : H^*(Y_{\emptyset}) &\to H^*(Y_{\set{1}}\times (D-0)) \\
  a_{11} & \mapsto a_{01} - g_{01} \\
  d_{11} &\mapsto f_{01} \\
  g_{11} &\mapsto g_{01} \\
  h_{11} &\mapsto h_{01} \\
  i_{11} &\mapsto i_{01} \\
  D_{11} &\mapsto d_{01} \\
  C_{11} &\mapsto a_{01} - g_{01} + y_{01,1} \\
  z_{11} &\mapsto y_{01,2} + (a_{01}-d_{01}+f_{01}-g_{01})y_{01,1}
\end{align*}
where $g_{01}$ is the generator of $H^1(D-0)$.

In degree 1, this is calculated by dualizing to homology, writing out
the map on generators and working
out what they map to. For $z_{11}$ it is slightly
harder. To calculate this, we use the submanifold $R \subset Z$
defined previously as
\[ R = \set{(x,y,z)\in \R^3 \delim \Big\vert\delim x > 0, y < 0, z =
  \frac{y-1}{1+xy}>0} \]
by choosing a generator $\alpha = [\sigma]$ of $H_2(Y_{\set{1}}\times
(D-0))$ and calculating the intersection of the image of $f = \pr_Z
\circ i\circ \sigma$ and $R$. Note that if the coordinates of $Y$ are
fixed at $(0,1)$, then the image of
\[ f(a,d,f,h,i,0,1,g) =
\left(0,\frac{1}{C}-\frac{a}{gC},-\frac{d}{fC} \right)\]
never intersects $R$ since $x = 0$. Likewise, the coordinates $i$ and
$h$ do not appear in the projection to $Z$ and can not influence the
result. Hence we only have to check the duals of the following
generators in degree 2:
\[ a_{01} y_{01,1}, d_{01} y_{01,1}, f_{01} y_{01,1}, y_{01,1} g_{01},
y_{01,2} \]
These will be handled individually. All non-relevant coordinates will
be set to 1, except $g$ which will be denoted $\e$ to make it clear
that this coordinate is small. If $y_{01,1}$ is involved, we choose the
generator
\[ \sigma(\lambda) = (1-\lambda,1) \qquad \lambda\in S^1 \]
as this gives
\[ y_{01,1} = [\lambda \mapsto 1-(1-\lambda)\cdot 1] = [\lambda
\mapsto \lambda] \]

\paragraph{$\mathbf{a_{01} y_{01,1}}$:}
The relevant map is
\[ (a,\lambda) \mapsto \left( (1-\l)C, \frac{1}{C}-\frac{a}{\e
    C},\frac{1}{C}\right)\]
where $C = -1-\frac{a\l}{\e}$. If this intersects $R$ then $C$ has
to be real and positive by looking at the last coordinate, which
forces $\l$ real and negative by the first coordinate and $a$ real and
positive by the second. Hence there is at most one intersection, at
$(1,-1)$:
\[ f(1,-1) = \left( \frac{2-2\e}{\e},-1,\frac{\e}{1-\e}\right) \]
By inspection, this satisfies the conditions and hence belongs to
$R$, so we conclude that the image intersects once. This gives the
following formula:
\[ \langle f^*(z_{11}), a_{01}y_{01,1} \rangle = \langle
z_{11},f_*(a_{01}y_{01,1})\rangle = \pm 1 \]
To check the sign, we need to check if this map preserves the
orientation of the torus in $R$. This is done by differentiating the
map to get a map of tangent spaces
\[ Tf : TT^2 \to TR \]
and checking that the determinant of this map is
positive. This has been done in Sage, see Appendix
\ref{sec:sageorientation} for the code.

\paragraph{$\mathbf{d_{01} y_{01,1}}$:}
In this case the map becomes
\[ (d,\l) \mapsto \left( \frac{(1-\l)C}{d},\frac{\e-1}{\e C},
  \frac{d}{C} \right) \]
and by a similar analysis, this intersects in exactly the
point $f(1,-1)$, which is the same as before. However, this time the
orientation is reversed.

\paragraph{$\mathbf{f_{01} y_{01,1}}$:}
This has the same point of intersection and preserves the orientation.

\paragraph{$\mathbf{y_{01,1}g_{01}}$:}
This also intersects and preserves the orientation.

\paragraph{$\mathbf{y_{01,2}}$:}
This is calculated by the intersection points of the fundamental class
of the torus in $Y$, which was mentioned earlier:
\[ \widehat T = \set{(b,c) \in \C^2 \mid \size{b}=\size{c} = 2}
\subset Y \]
The map becomes
\[ (b,c) \mapsto \left( bC, \frac{c}{C}-\frac{1}{\e C},\frac{c}{C}
\right) \]
with $C = -bc^2-\frac{1-bc}{\e}$. We will write $x$ for the first
coordinate, $y$ for the second and $z$ for the third. An analysis as
before gives $bc = xz > 0$ and $c = \frac{1}{\e(z-y)} \in
\R$. So both $b$ and $c$ must be real and $c$ should be positive. Then
$xy = bc - b\e^{-1}<0$ shows that $b$ must be positive since $bc$ is
positive. The only possible point of intersection is
\[ f(2,2) = \left( \frac{6-16\e}{\e},
  \frac{2\e-1}{3-8\e},\frac{2\e}{3-8\e} \right) \]
Since $\e$ is very small, this intersects $R$ and the orientation is
preserved.

\subsubsection{The inclusion $Y_{\set{2}}\times (D-0) 
  \hookrightarrow Y_\emptyset$}

The inclusion in coordinates is
\begin{align*}
  (\C^*)^5\times Y \times (D-0) \longrightarrow (&\C^*)^7\times Z \\
  (a,d,f,g,i,b,c,h) \mapsto 
  \bigg(g&,h,i,a,-\frac{d}{h},\\
  & D=\frac{d}{h}(1-bc), \\
  & C= -af-\frac{a^2bh}{d}+\frac{a^2bhf}{d}+\frac{a^2b^2ch}{d}, \\
  & \frac{bC}{D}, \frac{cd}{hC}-\frac{a}{C},
  \frac{cD}{C}-\frac{ahD}{dC}+\frac{afhD}{dC}+
  \frac{abchD}{dC}\bigg)
\end{align*}

We calculate as before and get the following on generators:
\begin{align*}
  i^* : H^*(Y_{\emptyset}) &\to H^*(Y_{\set{2}}\times (D-0)) \\
  a_{11} & \mapsto a_{10} \\
  d_{11} &\mapsto d_{10} - h_{10} \\
  g_{11} &\mapsto g_{10} \\
  h_{11} &\mapsto h_{10} \\
  i_{11} &\mapsto i_{10} \\
  D_{11} &\mapsto d_{10} + y_{10,1} - h_{10} \\
  C_{11} &\mapsto f_{10} + a_{10} \\
  z_{11} &\mapsto y_{10,2} + (a_{10}-d_{10}+f_{10}+h_{10})y_{10,1}
\end{align*}
where $h_{10}$ is the generator of $H^1(D-0)$.

The calculations have been left out, but are very similar to the
previous ones.

\subsection{Second column}

The differential on the second column is the boundary map
\[ \delta : H^q((Y_{\set{1}}\sqcup Y_{\set{2}})\times (D,D-0)) \cong
H^q(F_1,F_0) \to H^{q+1}(F_2,F_1) \] 

To calculate this map, consider the inclusion of triples
\[ Y_{\set{1,2}}\times (D_2,D_2-0,(D-0)\times (D-0)) \hookrightarrow
(F_2,F_1,F_0) \]
given by inserting a small disc, sphere or torus on the zero-entries
of $Y_{\set{1,2}}$. There are homeomorphisms of pairs,
\begin{align*}
  \big(D_2-0,(D-0)\times(D-0)\big) \cong \big(D\times (D-0) \sqcup
  (D-0)\times D, (D-0)\times(D-0)\big) \\
  \big(D_2,D_2-0\big) \cong \big(D\times D,D\times
  (D-0)\cup (D-0)\times D\big) =  (D,D-0)\times (D,D-0)
\end{align*}
These gives a commuting diagram,
\[ \xymatrix{ H^q(F_1,F_0) \ar[r] \ar[d] & H^{q+1}(F_2,F_1)
  \ar[d]^\cong \\ 
  H^q(Y_{\set{1,2}}\times (D_2-0,(D-0)\times (D-0))) \ar[r]
%  \ar@{=}[d] 
&
  H^{q+1}(Y_{\set{1,2}} \times (D_2,D_2-0)) %\ar@{=}[d] 
%\\
%  H^q(Y_{\set{1,2}}\times ((D,D-0)\times (D-0) \sqcup (D-0)\times
%  (D,D-0))) \ar[r] & H^{q+1}(Y_{\set{1,2}}\times (D,D-0)\times (D,D-0))
} \]
The marked isomorphism follows from excision and the left map is
induced by an inclusion. If we apply the
homeomorphism of pairs to the bottom of the diagram, the boundary map
at the bottom becomes the sum of the two boundary maps,
\[ \xymatrix@R=0.1em{ H^q(Y_{\set{1,2}}\times (D,D-0)\times (D-0))
  \ar[]!<15ex,0ex>;[rd]!<-20ex,0ex>^{\delta} \\
  & H^{q+1}(Y_{\set{1,2}}\times (D,D-0)\times (D,D-0)) \\
  H^q(Y_{\set{1,2}}\times (D-0)\times (D,D-0))
  \ar[]!<15ex,0ex>;[ur]!<-20ex,0ex>_\delta } \]
The boundary maps are given by the long exact sequence of the pair
$(D,D-0)$, so we only need to consider the inclusion on the left. By
choosing the discs sufficiently small and using the identification of
$H^q(F_1,F_0)$ defined in the first section, 
we see that the inclusion maps the space $Y_{\set{1,2}}\times
(D,D-0)\times (D-0)$ to $Y_{\set{1}}\times(D,D-0)$ and
$Y_{\set{1,2}}\times (D-0)\times (D,D-0)$ to $Y_{\set{2}}\times
(D,D-0)$. To get the differential, we must calculate the sum of these
two maps:
\begin{align*}
  i_1^* : H^q(Y_{\set{1}}\times (D,D-0)) &\to H^q(Y_{\set{1,2}}\times
  (D,D-0)\times (D-0)) \\
  i_2^* : H^q(Y_{\set{2}}\times (D,D-0)) &\to H^q(Y_{\set{1,2}}\times
  (D-0) \times (D,D-0))
\end{align*}

\subsubsection{The inclusion $Y_{\set{1,2}}\times (D,D-0) 
\times (D-0) \to Y_{\set{1}} \times (D,D-0)$}   

In coordinates, the map becomes
\begin{align*}
  (\C^*)^3 \times Y \times (D,D-0)\times (D-0) &\to (\C^*)^5 \times Y
  \times (D,D-0) \\
  (a,d,i,b,c,g,h) &\mapsto \left(a,i,h,\frac{d}{h},
  \frac{d}{h}(1-bc), \frac{bh}{d},\frac{dc}{h},g\right)
\end{align*}

Again, we calculate the map on cohomology by dualizing to homology
and working with the dual generators. To check what happens to the
elements of $H^q(Y)$, we count intersections with the subspace $R_Y =
\set{(b,c)\in \R^2 \mid b > 0, c > 0, bc > 1}$ of $Y$, exactly as for
$Z$ above.
\begin{align*}
  i_1^* : H^q(Y_{\set{1}}\times (D,D-0)) &\to H^q(Y_{\set{1,2}}\times
  (D,D-0)\times (D-0)) \\
  a_{01} &\mapsto a_{00} \\
  d_{01} &\mapsto d_{00}-h_{00} + y_{00,1} \\
  f_{01} &\mapsto d_{00} - h_{00} \\
  g_{01} &\mapsto g_{00} \\
  h_{01} &\mapsto h_{00} \\
  i_{01} &\mapsto i_{00} \\
  y_{01,1} &\mapsto y_{00,1} \\
  y_{01,2} &\mapsto y_{00,2} - d_{00}y_{00,1} + h_{00}y_{00,1}
\end{align*}
We denote the generator of $H^2(D,D-0)$ in
$H^*(Y_{\set{1}}\times (D,D-0))$ by $g_{01}$ and for
$H^*(Y_{\emptyset} \times (D,D-0) \times (D-0))$ we consider $g_{00}\in
H^2(D,D-0)$ and $h_{00} \in H^1(D-0)$. 

\subsubsection{The inclusion $Y_{\set{1,2}}\times (D-0)
\times (D,D-0) \to Y_{\set{2}}\times (D,D-0)$}
This is exactly as above for the map
\begin{align*}
  (\C^*)^3 \times Y \times (D-0)\times (D,D-0) &\to (\C^*)^5 \times
  Y\times (D,D-0) \\ 
  (a,d,i,b,c,g,h) &\mapsto
  \left(\frac{a}{g},d,i,g,1-bc,b,c,h\right)
\end{align*}
In cohomology, this becomes
\begin{align*}
  i_2^* : H^q(Y_{\set{2}}\times (D,D-0)) &\to H^q(Y_{\set{1,2}}\times
  (D-0) \times (D,D-0)) \\
  a_{10} &\mapsto a_{00} - g_{00} \\
  d_{10} &\mapsto d_{00} \\
  f_{10} &\mapsto y_{00,1}\\
  g_{10} &\mapsto g_{00} \\
  h_{10} &\mapsto h_{00} \\
  i_{10} & \mapsto i_{00} \\
  y_{10,1} &\mapsto y_{00,1} \\
  y_{10,2} &\mapsto y_{00,2}
\end{align*}
with $g_{01}\in H^2(D,D-0)$ inside $H^*(Y_{\set{2}}\times (D,D-0))$ and
$g_{00}\in H^2(D,D-0)$, $h_{00}\in H^1(D-0)$ inside
$H^*(Y_{\emptyset}\times(D-0)\times(D,D-0))$.

\subsubsection{Boundary maps}
\label{disk}
We have now reduced everything to calculating some well-known boundary
maps for the pair $(D,D-0)$. By writing out the long exact sequence,
the boundary map
\[ \delta : H^*(D-0) \to H^*(D,D-0) \]
is an isomorphism in degree 1,
\[ H^1(D-0) \cong H^2(D,D-0) \]
and zero in all other degrees. The boundary map for the pair $X\times
(D,D-0)$ is $\Id\otimes \delta$ in tensor-notation. So the boundary
map in the cases above picks out the ``new'' coordinate added to
$X\times (D-0)$ and throws away everything that does not depend on
it. To be more precise, the map is given on generators by:
\begin{align*}
  H^*(X)\otimes H^*(D-0)) &\to H^*(X)\otimes H^*(D,D-0)) \\
  x\otimes d &\mapsto
  \begin{cases}
    x \otimes \delta d & \text{if } d\in H^1(D-0) \\
    0 & \text{otherwise}
  \end{cases}
\end{align*}

\section{Turning the page}
\label{sec:rand}

We can now use the above to calculate the $E_2$-page. We have
calculated the groups appearing on the first page and the
differential. To get the second page, we need to compute the homology
groups of $d_1$. This reduces to linear algebra over the integers, and
is done using Sage. The result can be found in Figure \ref{fig:e2}.

\begin{figure}[ht]
  \[ \xy
  (5,4); (5,-80); **\dir{-}; (5,4); *\dir{>};
  (5,-80); (42,-80); **\dir{-}; *\dir{<};
  (0,-82)*{q}; (5,-84)*{p};
  (1,-84); (4,-81); **\dir{-};
  (0,0)\xymatrix@R=1em@C=1.7em{
    9 & \Z & 0 & 0 \\
    8 & \Z^5 & 0 & 0 \\
    7 & \Z^{11} & 0 & 0 \\
    6 & \Z^{15} & \Z & 0 \\
    5 & \Z^{16} & \Z^{3} & 0 \\
    4 & \Z^{16} & \Z^{3} & 0 \\
    3 & \Z^{15} & \Z & 0 \\
    2 & \Z^{11} & 0 & 0 \\
    1 & \Z^{5} & 0 & 0 \\
    0 & \Z & 0 & 0 \\
    & 0 & 1 & 2
  } \endxy \]
  \caption{The $E_2$-page.}
  \label{fig:e2}
\end{figure}

The differentials from $E_2$ onwards are given as maps:
\[ d_r : E_r^{p,q} \to E_r^{p+r,q+1-r} \]
For $r\geq 2$, this increases $p$ by at least 2. This means that
the domain or codomain of $d_r$ will always be zero, so taking
homology does not change anything. Hence $E_2 = E_\infty$, the
spectral sequence collapses at the 2\textsuperscript{nd} term, and
there are no extension problems since all the groups are free and
abelian. This allows us to read off the cohomology of $Y_{3,3}$. Since
we are also interested in factoring out the $T^3$ torus, we note this
as well.
\[ H^q(Y_{3,3}) = E_2^{0,q} \oplus E_2^{1,q-1} = 
\begin{cases}
  \Z & q = 0 \\
  \Z^{5} & q = 1\\
  \Z^{11} & q = 2\\
  \Z^{15} & q = 3\\
  \Z^{17} & q = 4\\
  \Z^{19} & q = 5\\
  \Z^{18} & q = 6\\
  \Z^{12} & q = 7\\
  \Z^{5} & q = 8\\
  \Z & q = 9\\
  0 & \text{otherwise} 
\end{cases} \qquad
H^q(X_{3,3}) =
\begin{cases}
  \Z & q = 0 \\
  \Z^{2} & q=1 \\
  \Z^{2} & q=2 \\
  \Z^{2} & q=3 \\
  \Z^{3} & q=4 \\
  \Z^{2} & q=5 \\
  \Z & q=6 \\
  0 & \text{otherwise}
\end{cases}
\]
We can also read off the ring structure from the spectral sequence, as
it is induced from the spaces we start with. Since we do not
need it, it will not be included here.
% Også noget med \pi_1? Eg. surjektivitet eller lignende?


%%% Local Variables: 
%%% mode: latex
%%% TeX-master: "main"
%%% End: 
